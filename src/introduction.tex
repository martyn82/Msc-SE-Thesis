\chapter{Introduction}
\label{introduction}

This chapter introduces the study by describing the problem this study is
addressing and the motivation for the research direction.

\section{Problem}
Many elements of the software development process have been found to be related
to the project's progress. These elements include, but are not limited to, team
composition, team size, frequency of releases, developer's activity, and
developer's and user satisfaction
\cite{crowston2006}\cite{delone1992}\cite{samoladas2010}.

A change in one or more of these elements affect the project's progress. The
effect of the change on the project's progress can be positive or negative. If a
change in one or more of the elements has a substantial effect on the
project's progress, we call it an \emph{evolutionary event}\rm; an event that
changed the project's evolution.

\paragraph{}
In closed source projects, many evolutionary events are the effect of decisions
made by the management of these projects. The majority of closed source
software projects are initiated and maintained by commercial organisations.
These projects evolve according to changing business requirements, strategies,
and commercial goals.

In these organisations, a change in the hiring policy, or planning and
deadlines, can become an evolutionary event in relation to the projects
affected by these changes. However, these events can be used to identify
project scale and progress. This allows comparing and co-analysing different
projects.

\paragraph{}
In open-source software (OSS) projects, evolutionary events are mostly
theoretical with little empirical validation. Changes in team size, and
activity, are organical. A community of an OSS project changes continuously,
often with little impact on the evolution of the project. This means that
different OSS projects cannot be compared on the same scale.

\paragraph{}
Comparing and analysing software projects is of interest to better understand
why some projects fail and others succeed. Up to today, it is hard to pinpoint
the key factors that define the success or failure of a software project.
Therefore, a means to identify the events that shape a project's evolution would
be useful. For closed-source software projects in a commercial environment,
these events can be easily tracked by the managerial history of the project. For
OSS projects, this is a different story.



\section{Motivation}
OSS projects are being used by an increasing number of (commercial)
organisations. Therefore, the evaluation of the survivability of OSS projects
becomes increasingly important.

\paragraph{}
At \hostOrg, developers extensively use OSS projects in their day-to-day
work. The projects vary from developers tools, and utilities, to operating
systems and full system stacks. These OSS projects include, but are not limited
to, Ansible, Apache web server, MySQL database, and Ubuntu/Linux. Besides the
previously listed well-known projects, lesser known projects are also used.

\paragraph{}
Given the increasing popularity of OSS projects, and the use of many OSS
projects by the host organisation, a means to assess OSS project's
survivability is useful. An assessment of the survivability of an OSS project
will aid in the decision for selection and/or continuity of a project.

\paragraph{}
In the study \emph{Automatic Means of Identifying Evolutionary Events in
Software Development }\rm by Siim Karus \cite{karus2013}, Karus proposed the
use of wavelet analysis to automatically detect anomalies in the evlution of
OSS projects.

\paragraph{}
Wavelet analysis is the analysis of waveforms. It is a tool that started in the
field of seismology, but is expanded to many other fields today. Economics,
geographics, audio, video, compression techniques, AD-convertors, and many kinds
of digital signal processing (DSP) is using wavelet transformations and
analysis.

A big advantage of wavelet analysis is that it is built upon solid mathematical
principles of waveforms, integrals, and statistics, making it fairly easy to
automate.

\paragraph{}
At first glance, it may seem strange to use wavelet analysis to analyse software
projects. Software evolution comprises various metrics, such as lines of code,
number of developers, etc. These metrics can be modeled in the frequency domain
of a waveform. A variable that has a functional relation to these
metrics is the \emph{age }\rm of a project, which naturally fits the time domain
of a waveform.

\paragraph{}
When analysing these waveforms, we should be able to detect anomalies in
these waveforms. After investigating and comparing these anomalies we might as
well be able to find the events that caused these evolutionary patterns. A
comparison with known events that had a negative effect on a project's progress
may reveal that we are able to detect warning signs in software evolution.\\

\noindent
Knowing these warning signs helps in:
\begin{itemize}
	\item Choosing an OSS project to implement in a business scenario.
	\item Making timely preparations for decommissioning an OSS project.
	\item Choosing a development process that best suits the aims of a software
	project for new OSS projects.
\end{itemize}

\paragraph{}
A \emph{warning sign }\rm is any pattern in the waveform that eventually results
in the \emph{end of code evolution }\rm of a project. The \emph{end of code
evolution }\rm means that the source code of a project does not change anymore.
For a pattern being such a warning sign, it has to be similar to patterns that
are confirmed to be such warning signs.

% \section{Replication}
% This study is a replication of the research by Siim Karus \emph{Automatic Means
% of Identifying Evolutionary Events in Software Development} \cite{karus2013}.
% 
% \subsection{The original study}
% The original study by Karus had the following research question:\\[0.5cm]
% \emph{Can we use wavelet analysis to find objective warning signs in
% software projects that lead to the end of code evolution?}\rm

\section{Research questions}
\label{questions}

The main research question (RQ1) is:\\

\begin{description}
	\item[RQ1] \emph{\researchQuestion}\\[0.3cm]
\end{description}

\noindent
In order to find the answer to the main question, the following
questions will be answered:
\begin{description}
	\item[RQ2] \emph{\subQuestionOne}
	\item[RQ3] \emph{\subQuestionTwo}
\end{description}

\section{Outline}

Chapter \ref{background} will provide more information on prior research in the
field of OSS project survivability. In chapter \ref{method} the method and
approach of the research is discussed. Chapter \ref{research} describes the
research execution. In chapter \ref{results}, the results are presented, and
chapter \ref{analysis} provides the results after analysis and presents the
conclusions of the research.

\begin{comment}

INTRODUCTION

An introduction and overview of your thesis. Be sure to structure your thesis
such that you do not have to repeat yourself later. In this section you do not
cover details, but you give the reader an idea of the context, a brief overview
of the research, and how the remainder of the thesis is structured.

\end{comment}
\begin{comment}

MOTIVATION

This section describes in detail what problem the research is addressing, and
what the motivation is to address this problem.

There is a concise and objective statement of the research questions (or
hypotheses you are testing) and goals. It is made clear why these questions and
goals are important and relevant to the outside world (i.e., the field of
research or industry that you are addressing). You can already split the main
research question into sub questions in this chapter.

This section also describes an analysis of the problem: where does it occur and
how, how often, and what are the consequences?

An important part is also to scope to research: what aspects are included and
what aspects are deliberately left out, and why? An example introduction can be
found on Paul Klint’s
homepage\footnote{http://homepages.cwi.nl/~paulk/thesesMasterSoftwareEngineering/2006/ReinierLabee.pdf}.
\end{comment}
