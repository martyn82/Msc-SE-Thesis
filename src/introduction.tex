\chapter{Introduction}
\label{introduction}

This chapter introduces the research topic by describing the problem the
research is aiming to solve, and the motivation for the research direction.

\section{Problem analysis}
Many elements of the software development process are related to the project's
progress. These elements include, but are not limited to, team composition,
team size, frequency of releases, developer's activity, and developer's and
user satisfaction \cite{crowston2006, delone1992, samoladas2010}.

A change in one or more of these elements affects the project's progress
positively or negatively. If a change in one or more of the elements has a
substantial effect on the project's progress, we call it an \emph{evolutionary
event}\rm; an event that changed the project's evolution.

\paragraph{}
In closed source projects, many evolutionary events are the effect of decisions
made by the management. The majority of closed source software projects are
initiated and maintained by commercial organisations. These projects evolve
according to changing business requirements, strategies, and commercial goals.

In commercial organisations, a change in the hiring policy, or planning and
deadlines, can become an evolutionary event for the projects affected by these
changes. These events can be used to identify project scale and progress. This
allows comparing and co-analysing different projects \cite{karus2013}.

\paragraph{}
In open-source software (OSS) projects, evolutionary events are mostly
theoretical with little empirical validation. Changes in development activity,
community, and other aspects of the OSS process are mostly organic. A community
of an OSS project changes continuously, often with little impact on the
evolution of the project \cite{androutsellis}. This means that different OSS
projects cannot be compared on the same scale. In other words, different OSS
projects cannot be objectively assessed by their evolution data without a
proper transformation of that data.

\paragraph{}
For closed source projects, the measurement of a project's success or
effectiveness is critical to our understanding of the value and efficacy of
management actions and investments \cite{delone2003}.

Comparing and analysing OSS projects is of interest to better understand the
survivability of OSS projects. The key factors that define the success or
failure of an OSS project have not yet been found.

\paragraph{}
We expect to be able to find indicators of failure in the events that
shape a project's evolution. Therefore, a means to identify these evolutionary
events would be useful. For closed-source software projects in a commercial
environment, these events can be easily tracked by the managerial history of
the project. For OSS projects, this is not the case.



\section{Motivation}
At \hostOrg, developers extensively use OSS projects in their day-to-day
work. The projects vary from developers tools, and utilities, to operating
systems and full system stacks. These OSS projects include, but are not limited
to, Ansible, Apache Web Server, MySQL, OpenStack, OpenVPN, and Ubuntu/Linux.
Besides the previously listed well-known projects, lesser known projects are
also used.

\paragraph{}
Given that OSS projects are popular in many organisations, and, as a result,
OSS projects contribute to the overall value of the products and services of
these organisations, a means to objectively assess OSS project's survivability
is of value. An assessment of the survivability of an OSS project will aid in
the decision for selection and/or continuity of a project.

\paragraph{}
In the study \emph{Automatic Means of Identifying Evolutionary Events in
Software Development }\rm by Siim Karus \cite{karus2013}, Karus proposed the
use of wavelet analysis to automatically detect anomalies in OSS projects.

\paragraph{}
Wavelet analysis is used in many fields having many purposes. One of the
applications of wavelet analysis is pattern recognition in digital signal
processing, such as duplication detection and face recognition in imagery
\cite{myna, wadkar}. The wavelet algorithms process data at different scales or
resolutions \cite{graps}. The big advantage of wavelet analysis is that we are
able to analyse the data at different scale levels. Another advantage is that it
is based upon mathematical principles, making it fairly easy to automate.

\paragraph{}
At first glance, it may seem odd to use wavelet analysis to analyse evolution of
software projects. Software evolution data comprises various metrics, such as
lines of code, number of developers, number of commits, etc. The measures of
these metrics are all snapshots of moments in time. These metrics can be
modeled in the frequency domain of a waveform. The metrics have a functional
relation to the age of a project, which fits the time domain of a waveform.
This gives a natural transition of each metric to be modeled as a waveform and
enables the analysis of such a waveform.

\paragraph{}
When analysing waveforms of software evolution metrics, we should be able to
detect anomalies in these waveforms. After investigating and comparing these
anomalies we might as well be able to find the patterns that resulted from
evolutionary events. A comparison with known events that had a negative effect
on a project's progress may reveal that we are able to detect warning signs in
software evolution.\\

\noindent
Knowing these warning signs will help in \cite{karus2013}:
\begin{itemize}
	\item Choosing an OSS project to implement in a business scenario.
	\item Making timely preparations for decommissioning an OSS project.
	\item Choosing a development process that best suits the aims of a software
	project for new OSS projects.
\end{itemize}

\paragraph{}
A \emph{warning sign }\rm is any pattern in the waveform that has a high chance
of resulting in the \emph{end of code evolution }\rm of a project. The \emph{end
of code evolution }\rm means that the source code activities of a project have
stopped. There could still be commits on the project, for instance in Wiki
pages, external library updates, or documentation, but no more source code
changes. For a pattern being such a warning sign, it has to be similar to
patterns that are confirmed to be such warning signs.



\section{Research questions}
\label{questions}

\begin{description}
	\item[RQ1] \emph{\researchQuestion}\\[0.3cm]
\end{description}

\noindent
In order to find the answer to the main question, the following
questions will be answered:
\begin{description}
	\item[RQ2] \emph{\subQuestionOne}
	\item[RQ3] \emph{\subQuestionTwo}
\end{description}

\paragraph{}
We are not looking for OSS vitality as the project's ability to provide support
and grow in the number of releases. In this study, we only look at the
development process using software analytics, ignoring documentation and
community support.

\section{Outline}

Chapter \ref{background} will provide more information on prior research in the
field of OSS project survivability. In chapter \ref{method} the method and
approach of the research is discussed. Chapter \ref{research} describes the
research execution. In chapter \ref{results}, the results are presented, and
chapter \ref{analysis} provides the results after analysis and presents the
conclusions of the research.

\begin{comment}

INTRODUCTION

An introduction and overview of your thesis. Be sure to structure your thesis
such that you do not have to repeat yourself later. In this section you do not
cover details, but you give the reader an idea of the context, a brief overview
of the research, and how the remainder of the thesis is structured.


MOTIVATION

This section describes in detail what problem the research is addressing, and
what the motivation is to address this problem.

There is a concise and objective statement of the research questions (or
hypotheses you are testing) and goals. It is made clear why these questions and
goals are important and relevant to the outside world (i.e., the field of
research or industry that you are addressing). You can already split the main
research question into sub questions in this chapter.

This section also describes an analysis of the problem: where does it occur and
how, how often, and what are the consequences?

An important part is also to scope to research: what aspects are included and
what aspects are deliberately left out, and why? An example introduction can be
found on Paul Klint’s
homepage\footnote{http://homepages.cwi.nl/~paulk/thesesMasterSoftwareEngineering/2006/ReinierLabee.pdf}.
\end{comment}
