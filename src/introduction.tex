\chapter{Introduction}
\label{introduction}

\section{Problem analysis}

In closed source projects, many evolutionary events are dictated by the
management.
Events such as team size changes, changes in activity and progress are often the
result of a change in, for instance, the hiring policy of the organization, or
an approaching deadline. These events can be used to identify project scale and
progress, which allows comparing and co-analyzing different projects.\\

In an open-source software (OSS) project, evolutionary events are mostly
theoretical with little empirical validation. Changes in team size, activity,
and the like are more organic. This means that different OSS projects cannot be
compared in the same scale.\\

OSS projects are being used by many organizations in an industrial environment.
This makes evaluating the survivability of OSS projects an important task.
At \theOrganization, developers extensively use OSS projects in their day-to-day
work. The projects vary from developers tools, and utilities, to operating
systems and full system stacks. These OSS projects include, but are not limited
to, Ansible, Apache web server, MySQL database, and Ubuntu/Linux. Amongst the
previously listed well-known projects, less known projects are used.

\paragraph{}
Automatically detecting warning signs in the evolution of OSS projects is useful
if a company is using OSS projects to support their business. This business can
be anything that is valuable to that company. The method can aid in selecting
OSS projects for new purposes. For instance, selecting a healthy project that
has a bigger chance of surviving.
Additionally, the method can be used to monitor projects that the people
at the company are using. In case anomalies occur, a decision can be made to
look for alternatives and eventually replace projects that have a lower chance
of survival.

\paragraph{}
This study is a replication and extension of the research by Siim Karus
\emph{Automatic Means of Identifying Evolutionary Events in Software
Development} \cite{karus2013}. It is extended by using a larger data set and
attempts to find warning signs in the software evolution of OSS projects.

\section{Research questions}
\label{questions}

The main research question of the thesis is (RQ1):
\begin{description}
	\item \hfill \\ \emph{\researchQuestion}
\end{description}

\paragraph{}
Wavelet analysis is the analysis of waveforms. It is an analysis tool that
started in seismology. Its use is expanded to many other fields, such as,
economics, geographics, audio and video processing, and others, to perform
all kinds of digital signal processing (DSP).

Software evolution comprises various measures, such as lines of code, number of
developers, etc. These metrics can be put in the frequency domain of a waveform,
whereas the age in months of a project can be put in the time domain of a
waveform. This way, wavelet analysis can be used to analyze software evolution.

As wavelet analysis is a tool that is relatively easy to automate, it
provides automatic detection of patterns in software evolution.

\paragraph{}
In order to find the answer to the main question, the following questions will
be answered:
\begin{description}
	\item \hfill \\ \emph{\subQuestionOne} (RQ2)
	\item \hfill \\ \emph{\subQuestionTwo} (RQ3)
\end{description}

\section{Outline}

Chapter \ref{background} will provide more background and context about the
research subject. In chapter \ref{method} the method and approach of the
research is discussed. Next, in chapter \ref{research}, the research execution
itself is described. Chapter \ref{results} presents the results of the research.
Chapter \ref{analysis} discusses the results after analysis and presents the
conclusions of the research.

\begin{comment}

INTRODUCTION

An introduction and overview of your thesis. Be sure to structure your thesis
such that you do not have to repeat yourself later. In this section you do not
cover details, but you give the reader an idea of the context, a brief overview
of the research, and how the remainder of the thesis is structured.

\end{comment}
\begin{comment}

MOTIVATION

This section describes in detail what problem the research is addressing, and
what the motivation is to address this problem.

There is a concise and objective statement of the research questions (or
hypotheses you are testing) and goals. It is made clear why these questions and
goals are important and relevant to the outside world (i.e., the field of
research or industry that you are addressing). You can already split the main
research question into sub questions in this chapter.

This section also describes an analysis of the problem: where does it occur and
how, how often, and what are the consequences?

An important part is also to scope to research: what aspects are included and
what aspects are deliberately left out, and why? An example introduction can be
found on Paul Klint’s
homepage\footnote{http://homepages.cwi.nl/~paulk/thesesMasterSoftwareEngineering/2006/ReinierLabee.pdf}.
\end{comment}
