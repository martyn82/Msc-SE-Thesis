\chapter{Introduction}
\label{introduction}

\section{Problem analysis}

In closed source projects, many evolutionary events are dictated by the
management.
Events such as team size changes, changes in activity and progress are often the
result of a change in, for instance, the hiring policy of the organization, or
an approaching deadline. These events can be used to identify project scale and
progress, which allows comparing and co-analyzing different projects.\\

The evolution of an open-source software (OSS) project is more organic. Events
that shape the OSS project's evolution are mostly theoretical with little
empirical validation. This means that different projects cannot be compared
in the same scale.\\

OSS projects are being used by many organizations in an industrial environment.
The number of organizations that use OSS projects is increasing. This makes
evaluating the survivability of OSS projects more important.\\

At \theOrganization, developers extensively use OSS projects in their day-to-day
work. The projects vary from developers tools, and utilities, to operating
systems and full system stacks. These OSS projects include, but are not limited
to, Ansible, Apache web server, MySQL database server and client, and Ubuntu
Linux.

Amongst the previously listed well-known projects, less known projects are used.
Therefore, it is valuable to be able to determine or even predict the
survivability of these projects. The detection of warning signs in and the
monitoring of their evolution is useful to decide whether to choose for or
continue to use an OSS project.\\

This study is a replication of the research described in the paper
\emph{Automatic Means of Identifying Evolutionary Events in Software
Development} by Siim Karus. In this study, the evolution data of 250 OSS
projects is analyzed to find patterns that can be interpreted as warning
signs.

\section{Research questions}
\label{questions}

The main research question (RQ1) of the thesis is:
\begin{description}
	\item \hfill \\ \emph{\researchQuestion}
\end{description}

\noindent\\
In order to find the answer to the main question, the following questions will
be answered:
\begin{description}
	\item[RQ2] \hfill \\ \emph{\subQuestionOne}
	\item[RQ3] \hfill \\ \emph{\subQuestionTwo}
\end{description}

\section{Outline}

Chapter \ref{background} will provide more background and context about the
research subject. In chapter \ref{method} the method and approach of the
research is discussed. Next, in chapter \ref{research}, the research execution
itself is described. Chapter \ref{results} presents the results of the research.
Chapter \ref{analysis} discusses the results after analysis and presents the
conclusions of the research.

\begin{comment}

INTRODUCTION

An introduction and overview of your thesis. Be sure to structure your thesis
such that you do not have to repeat yourself later. In this section you do not
cover details, but you give the reader an idea of the context, a brief overview
of the research, and how the remainder of the thesis is structured.

\end{comment}
\begin{comment}

MOTIVATION

This section describes in detail what problem the research is addressing, and
what the motivation is to address this problem.

There is a concise and objective statement of the research questions (or
hypotheses you are testing) and goals. It is made clear why these questions and
goals are important and relevant to the outside world (i.e., the field of
research or industry that you are addressing). You can already split the main
research question into sub questions in this chapter.

This section also describes an analysis of the problem: where does it occur and
how, how often, and what are the consequences?

An important part is also to scope to research: what aspects are included and
what aspects are deliberately left out, and why? An example introduction can be
found on Paul Klint’s
homepage\footnote{http://homepages.cwi.nl/~paulk/thesesMasterSoftwareEngineering/2006/ReinierLabee.pdf}.
\end{comment}
