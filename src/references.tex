\addcontentsline{toc}{chapter}{Bibliography}
\begin{thebibliography}{99}

	\bibitem{androutsellis} \sc{S. Andoutsellis-Theotokis, D. Spinellis, M.
		Kechagia, G. Gousios}\rm, Open Source Software: A Survey from 10,000
		Feet, \emph{Technology, Informations and Operations Management, vol. 4, Nos.
		3-4, 2010, p187-347}\rm

\begin{comment}
		This 'paper' of 160 pages gives a wide overview of the OSS eco-system. It
		provides insight in the history and how and why the OSS community works, who
		is involved, and the impact on the software community in general. It even
		gives insight in the business models and adoption by the industry. The
		information gives a solid background to the Master's project to understand
		the results found by the study.
\end{comment}
		
	\bibitem{bru2013} \sc{M. Bruntink}\rm, Towards Base Rates in Software
		Analytics: Early Results and Challenges from Studying Ohloh, \emph{Science of
		Computer Programming, Oct. 2013}\rm

\begin{comment}
		The data set that was collected here is a subset of the ohloh universe using
		validation techniques as described by Nagappan et al.
\end{comment}

	\bibitem{ohlohanalytics} \sc{M. Bruntink}\rm, OhlohAnalytics data set
		and analysis tools, \emph{http://github.com/MagielBruntink/OhlohAnalytics,
		University of Amsterdam, 2013}\rm

	\bibitem{crowston2003} \sc{K. Crowston, H. Annabi, and J. Howison}\rm,
		Defining Open Source Software Success, \emph{School of Information Studies,
		24th International Conference on Information Systems, 2003}\rm

\begin{comment}
		Crowston et al. identified a range of measures that can be used to assess the
		success of OSS projects. They identified measures based on a review of the
		literature, a consideration of the OSS development process, and an analysis of
		the opinions of OSS developers. For each measure, they provided examples of
		how they might be used in a study of OSS development.

		The relation with this Master's project is to be able to measure the success
		of an OSS project. Some measures from the paper cannot be used in this study
		because they require personal interaction with inidividual developers.
		However, some other measures are useful. Such as the number of developers, the
		individual level of activity, and cycle time.
\end{comment}

	\bibitem{crowston2006} \sc{K. Crowston, J. Howison, and H. Annabi}\rm,
		Success in FLOSS Development: Theory and Measures, \emph{Software Process
		Improvement and Practice, 2006, no. 11, p123-148}\rm

\begin{comment}
		Crowston et al. studied the success of FLOSS projects. They reconsider what
		success means in a FLOSS context. They elaborated on the literature study
		conducted in \cite{crowston2003} and extended this conclusions by a critical
		evaluation on the measures defined therein.
		
		These new insights are useful in defining metrics to determine a project's
		health and activity.
\end{comment}

	\bibitem{delone1992} \sc{W.H. Delone and E.R. McLean}\rm, Information Systems
		Success: The Quest for the Dependent Variable, \emph{Information Systems
		Research, no. 3, 1992, p60-95}
	
	\bibitem{karus2013} \sc{S. Karus}\rm, Automatic Means of Identifying
		Evolutionary Events in Software Development, \emph{1063-6773/13 IEEE, 2013}\rm

\begin{comment}
		This is the base paper of the Master's project.
		Karus has explored the usage of wavelet transforms for the analysis of OSS
		projects evolution. Karus was able to detect close to 1,000 similar patterns
		in a subset of 27 OSS projects. In conclusion, the analysis reveals that
		wavelet analysis can be a powerful and objective tool for identifying
		evolutionary events that can be used as estimation basis or management guide
		in software projects.

		During this Master's project I will replicate and extend the work of Karus by
		comparing patterns and events in order to validate wavelet analysis as a tool
		to predict end of software evolution. This will be done using a larger data
		set and a critical validation of the results found in the base paper.
\end{comment}

	\bibitem{dumas} \sc{S. Karus and M. Dumas}\rm, Code Churn Estimation
		Using Organisational and Code Metrics: An Experimental Comparison,
		\emph{Institute of Computer Science Journal vol. 54, issue 2, 2012}\rm

\begin{comment}
		Karus and Dumas studied a means to estimate future code maintenance effort in
		order to plan software maintenance activities. In this paper they used machine
		learning techniques to unveil predictors of yearly cumulative code churn of
		software projects on the basis of metrics extracted from version control
		systems. They have shown that a code churn estimation model built purely with
		organisational metrics is superior to one built purely with code metrics.
		However, a combined model provides the highest predictive power. Code metrics
		in general are complementary to organisational metrics for the purpose of
		estimating code churn.

		The paper is related to the project in ways of determining future code churn.
		It will be of use to predict software evolution events.
\end{comment}

	\bibitem{elbaum} \sc{S.G. Elbaum and J.C. Munson}\rm, Code Churn: A
		Measure for Estimating the Impact of Code Change, \emph{IEEE Software
		Maintenance, 1998}\rm

\begin{comment}
		Elbaum and Munson present a methodology that, as they call it, will produce a
		viable fault surrogate. The measure for estimating the impact of code change
		is of interest in this paper. The authors introduced the notion of \emph{code
		churn}. A code churn is the total lines of code added, modified, and removed.
		It is a precise measurement of software development process and product
		outcomes.

		Code churn is an objective measure of software development progress. It
		measures the total work done. Counting the lines of code (LOC) is not enough
		as there could have been a refactoring of existing code that removed 20\% of
		the lines. With counting lines you would miss the work done to remove that
		code. Therefore, code churn is a better way to measure the development
		progress.
\end{comment}

	\bibitem{nagappan} \sc{M. Nagappan, T. Zimmermann, and C. Bird}\rm,
		Diversity in Software Engineering Research, \emph{ESEC/FSE'13, August 18–26,
		2013, Saint Petersburg, Russia, ACM 978-1-4503-2237-9}\rm

\begin{comment}
		One of the goals of software engineering research is to achieve generality:
		are the phenomena found in a few projects reflective of others? While it is
		common sense to select a sample that is representative of a population, the
		importance of diversity is often overlooked, yet as important. In this paper
		the authors present a measure called \emph{sample coverage}, defined as the
		percentage of projects in a population similar to the given example.
		
		The paper provides insight in why diversity is evenly important as
		representativeness. It provides computation techniques for evaluating a data
		set of software projects. The example computations are based on the Ohloh.net
		universe which is the same as in this Master's project.
\end{comment}

	\bibitem{raja2012} \sc{U. Raja and M.J. Tretter}\rm, Defining and
		Evaluating a Measure of Open Source Project Survivability, \emph{0098-5589/12
		IEEE Transactions on Software Engineering vol.38 no.1, jan-feb 2012}\rm

\begin{comment}
		Raja and Tretter defined and validated a multidimensional measure of OSS
		project survivability: project viability. This measure has three dimensions:
		vigor, resilience, and organisation. For each of these dimensions they defined
		a viability index. The outcome of the study is that they have been able to
		confirm that vigor, resilience, and organisation are different attributes
		related to project survivability.

		This paper is related to the Master's project in terms of determining whether
		an OSS project is active or dying.
\end{comment}

	\bibitem{samoladas2010} \sc{I. Samoladas, L. Angelis, I. Stamelos}\rm,
		Survival Analysis on the Duration of Open Source Projects, \emph{Information
		and Software Technology 52, 2012, p.902-922}\rm

\begin{comment}
		Samoladas et al. studied survival analysis techniques for estimating the
		future development of a FLOSS project. They used duration data of thousands of
		FLOSS projects from different source code repositories. They have applied
		different survival analysis methods to predict the survivability of the
		projects (i.e., their probability of continuation in the future). They have
		used the application domain and number of committers as metrics that affect
		duration.

		This paper is related to this project on the analysis of project survivability
		measures. It can be used to validate the results found by wavelet analysis.
\end{comment}
	
	\bibitem{wang2012} \sc{J. Wang}\rm, Survival Factors for Free Open
		Source Software Projects: A Multi-stage Perspective, \emph{European Management
		Journal 2012, no. 30, p352-371}\rm

\begin{comment}
		Wang uses a large data set of free OSS projects (FOSS) obtained from
		SourceForge.net to investigate survival factors at two stages of the project
		life-cycle. The first stage being the initial stage, before the first release.
		The second stage is the growth stage, after the first release. The factors for
		survival are slightly different in each stage. Wang has found that there are
		in general many factors that influence survivability of FOSS projects. Six
		factors are of particular interest: developer participation effort, developer
		service quality, software license restrictiveness, targeted users, community
		social network ties, and community quality of social ties. These six were
		chosen as these have the most influence on FOSS project success and the data
		of these factors is widely available.
		
		The results have shown that high quality developers who are devoted to
		participating in the project development activity and providing quality
		service to the users are of high importance for FOSS project success. Projects
		targeted at technical users have a higher likelihood of surviving at both
		stages. Furthermore, as user/developer participation, service quality,
		internalt and external network size, and quality of external network have been
		found to reliably predict survival at the initial stage, a weak showing in
		these measures alerts early signs of trouble.
\end{comment}

\end{thebibliography}
