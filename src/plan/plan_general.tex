\section{General information}
\subsection{Title}
\theTitle

\subsection{Student name}
Martijn Florian Endenburg B ICT \\
Student number: 10442677

\subsection{Host organisation and group}
VNU Vacature Media (De Persgroep Nederland) \\
Department of Technology, Development Team \\
mt. Lincolnweg 40 \\
1033 SN Amsterdam \\
The Netherlands \\
\href{http://www.vnuvacaturemedia.nl}{www.vnuvacaturemedia.nl}

\subsection{Contact person}
Rutger Pannekoek \\
CTO at VNU Vacature Media \\
phone: 020 - 204 25 21 \\
email:
\href{mailto:r.pannekoek@vnuvacaturemedia.nl}{r.pannekoek@vnuvacaturemedia.nl}

\subsection{Summary}
\paragraph{}
The Master's project will be a replication of the paper \emph{``Automatic Means
of Identifying Evolutionary Events in Software Development''} by Siim Karus
(2013) \cite{karus2013}. In this study by Karus he explored the usage of wavelet
analysis to identify evolutionary events (i.e., common patterns) in the
evolution of open source software (OSS) projects. The analysis was performed on
a data set of 27 OSS projects. This data set was chosen carefully to be a
representative set following the distribution of all OSS projects tracked by
Ohloh.net at the time of the study.
\\

During this Master's project I will replicate and validate the results found by
Karus. The replication will be done using a data set that is approximately 10 times
larger. At the time of this writing, ohloh.net tracks over 660,000 OSS projects.
The evolution data for these projects is openly available. Therefore, it should
be possible to have a data set of at least 250 OSS projects for this study.
\\

I will extend Karus' study by looking into the kinds of events that can be
detected. For this I will dive into the detected events and find out what they
mean in order to tell if an event is an objective warning sign. According to
Karus, knowing warning signs could help in:
\begin{itemize}
	\item Choosing an OSS project to implement in a business scenario.
	\item Making timely preparations for decommissioning an OSS project.
	\item Choosing a development process that best suites the aims of a software
	project for new OSS projects.
\end{itemize}

Additionally, I will test under which conditions wavelet analysis can be a
useful tool in detecting warning signs in software evolution, and which events
it is not able to detect.
\\

\noindent
\bfseries{The questions to be answered}\rm
\begin{description}
	\item[RQ:] \emph{\researchQuestion} \cite{karus2013}
	\item[Q1:] \emph{\subQuestionOne}
	\item[Q2:] \emph{\subQuestionTwo}
\end{description}
