\section{Problem analysis}
\paragraph{}
In closed source software projects, many evolutionary events are dictated by the
management. These events can be used to identify project scale and progress,
which allows comparing or co-analyzing different projects.

Unfortunately, the evolutionary events in OSS projects are mostly theoretical
with little empirical validation. This also means that different projects can
not be compared in the same scale.

In this study, we are proposing means for detecting evolutionary events in OSS
projects to offer comparable data on their evolution \cite{karus2013}. We use
wavelet analysis on time series in software evolution to be able to detect
events that may lead to the end of code evolution. We dive deeper into these
events to be able to find warning signs as early as possible in the evolution
of OSS projects.

\paragraph{}
At VNU Vacature Media developers extensively use OSS projects in their
day-to-day work. The projects vary from developers tools to complete subsystems.
These OSS projects include, but not limited to, Ansible, Apache web server, Zend
Framework, and MySQL database server.

Amongst the previous well-known OSS projects, less known OSS projects are used.
Therefore, it is valuable to be able to detect warning signs in the
evolution of such projects. The detection and monitoring of their evolution is
useful to decide whether to choose for and/or continue to use an OSS project.

\paragraph{}
The survivability of OSS projects has been of interest by many researchers. In
an industrial environment, a project is considered successful if it is completed
within time and budget constraints. In OSS projects, these constraints do not
always exist. The indicators of success for OSS projects differ from the one
study to the other.

In a study by Samoladas et al. a method for survival analysis on OSS projects
was proposed. These methods were employed to predict the
survivability of the projects by examining their duration, combined with other
characterstics such as application domain and number of committers. Although
these metrics give insight in the survivability chances of a project, it was
also found that adding a developer to the team of contributors increased the
survivability of the project substantially. The authors of the paper proposed
two main research issues to be addressed in the future. The first one is to add
more projects to the study with possibly a different categorization. In
addition, the effects of more project parameters, such as programming language
should be examined. This is not trivial since typically more than one language
is used in each project \cite{samoladas2010}.

\paragraph{}
There has been a study by Raja and Tretter on defining a measure of OSS project
survivability. They have been looking for vitality of OSS projects: the ability
of a project to grow and maintain its structure in the presence of
perturbations. They identified three dimensions of project viability: vigor --
the ability of a project to grow --, resilience -- the ability of a project to
recover from disturbances --, and organization -- the structure exhibited in the
project. These dimensions represent three distinct characteristics of project
viability \cite{raja2012}. Although this study is of use to determine
survivability of a project it is a snapshot of a point in time, whereas the
analysis of time series may provide a more objective judgement of project
survivability.

\paragraph{}
Crowston et al. identified measures that can be applied to assess the success of
OSS projects. The authors used the DeLone and McLean model of information
systems success to evaluate OSS project success. The aspects identified by
DeLone and McLean are elaborated; output of systems development -- it is
believed that a project that has a high frequency of releases is healthy --,
process of systems development -- the number of developers, the individual
level of activity, and cycle time (time between releases) --, and
project effects -- employment opportunities of the contributors, individual
reputation, and knowledge creation. It was found that most of these aspects are
indicators of OSS project success \cite{crowston2003}.

\paragraph{}
Another study conducted by Crowston et al. extended the previous study for
FLOSS projects. They found that the number of developers as a simple count of
developers is a bit of a flawed number as it aggregates the number of
developers leaving and the number of developers joining a project. A 'churn' of
the developers or a 'tenure' of individuals would be more appropriate.
Furthermore, this study took projects from one source code repository,
SourceForge, instead of different repositories \cite{crowston2006}.

\paragraph{}
A study conducted by J. Wang has shown that early warning signs can be found in
six crucial factors of OSS projects success: developer participation effort, developer
service quality, software license restrictiveness, targeted users, community
social network ties, and community quality of social ties \cite{wang2012}.
However, warning signs in these factors are not easily detected automatically.
