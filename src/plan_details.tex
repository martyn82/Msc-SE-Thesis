\section{Project details}

\subsection{Problem analysis}
\paragraph{}
In closed source projects, many evolutionary events are dictated by the
management. These events can be used to identify project scale and progress,
which allows comparing or co-analysing different projects.

Unfortunately, the evolutionary events in open source software projects are
mostly theoretical with little empirical validation. This also means that
different projects can not be compared in the same scale.

In this study, we are proposing means for detecting evolutionary events in open
source software projects to offer comparable data on their evolution
\cite{karus2013}.

\paragraph{}
\bf{Note: }\rm \emph{The previous seems an assumption by Karus as there is no
citation or study referred to that had found that the evolutionary events in closed source
software projects are mostly dictated by management. So, is this true?}

\bf{Note: }\rm \emph{Can or cannot we generalize the findings of this study to
both open source and industrial/closed source software projects?}

\subsection{Research method}
RQ: Can we use wavelet analysis to find objective warning signs in software
projects leading to the end of code evolution? \cite{karus2013}

\paragraph{}
According to Karus, knowing warning signs could help in:
\begin{itemize}
	\item Choosing OSS to implement in business scenario.
	\item Making timely preparations for decommissioning an OSS project.
	\item Choosing development process that best suites the aims of a software
	project for new OSS projects.
\end{itemize}

\paragraph{}
The research question is based on a theory how the chosen analysis method could
solve the question. A confirmatory case study will be done by choosing a
representative data set of open source software projects and analyse these by
using wavelet transforms to find evolutionary events.

The theory is based on the observations on analysis on time series in
economical and social studies. Which have been successfully done using wavelet
analysis. In the evolution of software projects we are also looking at time
series. Therefore, wavelet analysis seems a logical method to analyse software
project evolution.

\subsection{Expected results}
At the end of the research, we expect to know:
\begin{itemize}
	\item In which cases wavelet analysis can be used to find objective warning
	signs leading to end of code evolution.
	\item Which events can be detected by wavelet analysis and in which stage
	(early warning signs).
	\item Early detection of anomalies to take action before an anomaly becomes a
	long-term issue that might lead to the end of (part of) the project.
\end{itemize}

\subsection{Required expertise}
\begin{itemize}
	\item Expertise on discrete wavelet transformation and analysis.
	\item Mining source code repositories.
\end{itemize}

\subsection{Timeline}
\begin{description}
	\item[January 2014] \hfill \\ Project planning and literature study
	\item[February 2014] \hfill \\ Setup metrics
	\item[March 2014] \hfill \\ Building data set
	\item[April 2014] \hfill \\ Data gathering and analysis
	\item[May 2014] \hfill \\ Analysis and conclusion
	\item[June 2014] \hfill \\ Finish conclusion and present results
\end{description}

\subsection{Risks}
\begin{itemize}
	\item Improper understanding of wavelet transformation and analysis.
	\item Insufficient data to construct a data set that is representative to the
			real-world distribution of open source software projects in terms of scale,
			progress, and activity.
\end{itemize}