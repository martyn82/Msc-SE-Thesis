\chapter{Report on replication}

\section{About the original study}
The research question of the original study ``\emph{Automatic means of
identifying evolutionary events in software development}\rm'' conducted by
\citet{karus2013} in 2013, was ``\emph{Can we use wavelet analysis to find
objective warning signs in software projects leading to the end of code
evolution?}\rm''.

\subsection{Data set}
\citeauthor{karus2013} used a data set consisting of 27 OSS projects, of which
18 were randomly chosen using Google Code Search from different repositories
having different team sizes, main programming languages, and project types. 15
of these projects were on-going, and 3 had no activity since January 2009
(verified in January 2013) (i.e., 'dead').

\subsection{Design}
The research was done in the following steps:
\begin{enumerate}
	\item \citeauthor{karus2013} used weekly commit data to perform analysis on.
		Therefore, the first step was to aggregate daily commit data into weekly data.
	\item Then, the project signals were used as input for discrete wavelet
		transform using the Haar filter.
	\item Finding similar sequences.
	\item Similar sequence grouping (i.e., identifying patterns).
\end{enumerate}

\noindent
Table \ref{table:karus_series} shows the signals that were used for analysis in
the original study.

\begin{table}[H]
\caption{Project signals in the original study}\label{table:karus_series}
\centering
\begin{tabular}{ll}
	\hline
	\bfseries{Time}\rm & \bfseries{Frequency}\rm \\
	\hline
	Date & Active developers \\
	& Cumulative developers \\
	& Cumulative LOC added \\
	& Cumulative LOC churn \\
	& Cumulative LOC modified \\
	& Cumulative LOC removed \\
	& LOC \\
	& Relative date progress \\
	& Relative LOC churn \\
	& Relative team size \\
	& Files \\
	\hline
\end{tabular}
\hspace{1em}
\begin{tabular}{ll}
	\hline
	\bfseries{Time}\rm & \bfseries{Frequency}\rm \\
	\hline
	Cumulative LOC churn & Active developers \\
	& Cumulative developers \\
	& Cumulative LOC added \\
	& Cumulative LOC modified \\
	& Cumulative LOC removed \\
	& LOC \\
	& Relative date progress \\
	& Relative LOC churn \\
	& Relative team size \\
	& Files \\
	& Commits \\
	\hline
\end{tabular}
\end{table}

\subsection{Results}
998 common patterns were found across different projects. It showed that
``evolution patterns in software projects do express various warning signs
leading to or indicating the impending end of code evolution.''
\cite{karus2013}.
\citeauthor{karus2013} concluded, that: ``\ldots wavelet analysis can be
successfully used to detect anomalies before they turn into a long-term issue
allowing action to be taken before it is too late.'', and: ``In conclusion, we
have demonstrated the usefulness for both tracing software evolution and for
detecting anomalies in software evolution.''.

\section{About the replication}
\subsection{Level of interaction}
During this study, and especially in the beginning, there was contact with
\citeauthor{karus2013} about how the implementation of the wavelet transform and
detection of patterns was done. In that period, he provided his R scripts that
he used for his study.

\subsection{Changes}
A few modifications were made to the original study in replication.
\begin{description}
	\item[Sample size.] The sample size of 27 projects was extended to a total of
	250 projects.
	
	\item[Sample selection.] The selection of the projects for the study was done
	using a tool measuring the representativeness of the sample against a master
	set, to have a set that is as representative as possible.
	\citeauthor{karus2013} picked random projects and tested the set for
	representativeness.
	
	\item[Time series.] In the original study, daily commit data was aggregated
	into weekly data. In this study, we used montly data. Project data available at
	Ohloh is already aggregated into monthly data. We expect the difference between
	weekly and monthly data is negligible when looking for warning signs that may
	lead to the end of code evolution. The effects of patterns of such kind will be
	visible beyond the weekly boundaries.
	
	\item[Project signals.] The signals used by \citeauthor{karus2013} are listed
	in Table \ref{table:karus_series}. In the replicative study we picked only one
	series: the LOC over time. The choice for having just one signal is because of
	time constraints. The choice for LOC was because it can be interpreted
	intuitively.
\end{description}

\subsection{Results}
The results that came out of this study are comparable to those in the original
study. We have also found patterns in software evolution by using wavelet
analysis. Therefore, we can agree that ``\ldots evolution patterns in software
projects do express various warning signs leading to or indicating the impending end of
code evolution.'', but care should be taken. Of course not \emph{all }\rm
patterns detected are warning signs.

\paragraph{}
This study takes the results obtained by replicating the research by
\citeauthor{karus2013} further by looking into the patterns found. We have
identified types of patterns which have different semantics in the search for
patterns that are objective warning signs.