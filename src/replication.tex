\chapter{Replication report}
\label{replication}
This appendix reports on the replication using guidelines proposed in the
paper by \citet{carver}.

\section*{About the original study}
The original study, ``\repltitle{}'', was conducted by \replauthor{} in
\citeyear{karus2013}. The main research question of the original study is:

``Can we use wavelet analysis to find objective warning signs in software
projects leading to the end of code evolution?''

\subsection*{Data}
\citeauthor{karus2013} used a data set consisting of 27 OSS projects.

18 of the projects were randomly chosen using Google Code Search and contained
evolution data from different repositories. The projects vary in team sizes,
main programming languages, and project types. 15 of the projects were alive,
and 3 had no activity since January 2009 and were considered 'dead'. The
evaluation for dead or alive was done in January 2013.

The data set was then complemented with 9 dead projects from
SourceForge\footnote{\url{http://www.sourceforge.net}} in order to balance the
number of dead and alive projects. Making a total number of 27 projects.

\paragraph{}
\citeauthor{karus2013} used a specially built tool for mining the software
repositories. As a result, complete commit logs of the projects were gathered.

The construction of the project signals was done using the metrics shown in
Table \ref{table:karus_series}. \citeauthor{karus2013} used two kinds of
time-series dimensions: days since first commit, and cumulative code churn. A
total number of 20 types of signals were constructed for 27 projects, making a
total of 540 signals analysed in the original study.

\newcommand{\tableHeadSeries}{\bfseries{Time}\rm & \bfseries{Frequency}\rm}
\begin{table}[H]
\caption{Project signals in the original study}\label{table:karus_series}
\centering
\begin{tabular}{ll}
	\hline
	\tableHeadSeries \\
	\hline
	Date & Active developers \\
	& Cumulative developers \\
	& Cumulative LOC added \\
	& Cumulative LOC churn \\
	& Cumulative LOC modified \\
	& Cumulative LOC removed \\
	& LOC \\
	& Relative date progress \\
	& Relative LOC churn \\
	& Relative team size \\
	& Files \\
	\hline
\end{tabular}
\hspace{1em}
\begin{tabular}{ll}
	\hline
	\tableHeadSeries \\
	\hline
	Cumulative LOC churn & Active developers \\
	& Cumulative developers \\
	& Cumulative LOC added \\
	& Cumulative LOC modified \\
	& Cumulative LOC removed \\
	& LOC \\
	& Relative date progress \\
	& Relative LOC churn \\
	& Relative team size \\
	& Files \\
	& Commits \\
	\hline
\end{tabular}
\end{table}

\subsection*{Method}
The original study was done in the following steps:
\begin{description}
	\item[Data preparation.] The evolution data of the 27 projects was aggregated
		along the time-series dimensions. \citeauthor{karus2013} had daily commit
		data for all 27 projects. The 'days since first commit' metric was aggregated
		into 7 day-frames (i.e., weekly data), and the time-series of 'cumulative LOC
		churn' metric was aggregated into 1,000 LOC frames.
		
		The aggregation method for each of the metrics are shown in Table
		\ref{table:karus_series}.
		
	\item[Discrete wavelet transform.] The second step was applying the discrete
		wavelet transform with the Haar filter (Daubechies-2) on the data series.
		This step resulted in wavelet and filter coefficients for each compression
		level.
		
	\item[Similar region detection.] In the third step, the series of coefficients
		are compared to find linearly positively similar regions of sub-sequences. In
		the paper, \citet{karus2013} showed the method for defining what is similar.

		The similar sequences occurring at least 3 times are grouped together as a
		'pattern'.
\end{description}

\subsection*{Results}
\citet{karus2013} found 998 common patterns across different projects. He
confirmed that ``\textit{\ldots{}evolution patterns in software projects do
express various warning signs leading to or indicating the impending end of
code evolution.}''.

\paragraph{}
\citeauthor{karus2013} concluded that: ``\textit{\ldots{}wavelet analysis can be
successfully used to detect anomalies} [in software evolution] \textit{before
they turn into a long-term issue, allowing action to be taken before it is too
late.}'', and, ``\textit{In conclusion, we have demonstrated the usefulness
for both tracing software evolution and for detecting anomalies in software
evolution.}''.

\section*{About the replication}
\subsection*{Level of interaction}
During this study, and especially in the beginning, I was in contact with
\citeauthor{karus2013} about the implementation of his study. He explained some
of the decisions he made and the 'problems' he had conducting the study.

\paragraph{}
There are many knobs that can be tweaked in this kind of research. Therefore, to
be able to  fully replicate the results, \citeauthor{karus2013} provided his
raw data set and his analysis scripts. The three R scripts he had used for
detecting the patterns:
\begin{itemize}
	\item wavelet analysis: takes the data as input, and outputs
		the coefficients;

	\item similar sequence detection: takes the coefficients and
		outputs pairs of similar regions;

	\item sequence grouping: takes the pairs of similar regions and outputs the
		patterns.
\end{itemize}

\subsection*{Changes}
Table \ref{table:changes} presents what aspects of the original study were
changed in the replication. Each change is compared and motivated.

\begin{table}[H]
\caption{Changes to the original study}\label{table:changes}
\centering
\begin{tabular}{p{20mm}p{30mm}p{30mm}p{6cm}}
\hline
	\bfseries{Aspect}\rm & \bfseries{Original}\rm & \bfseries{Replication}\rm &
	\bfseries{Motivation}\rm \\
	\hline
	\bfseries{Data}\rm \\
	\hline
	Sample size & 27 projects & 250 projects & A larger data is believed to yield
	more trustworthy results in terms of the ability to generalise results. \\

	Sample selection & Random and manual complementing & Automatic
	representative selection & Knowing the representativeness and selecting
	projects accordingly increases the ability to generalise the results. \\

	Data gathering & Repository mining & Ohloh analysis & The data gathered by
	the tool \cite{ohlohanalytics, bruntink2013} was gathered from Ohloh
	\cite{ohloh}. Ohloh did the source code repository mining and aggregated and
	analysed this data into monthly facts. Furthermore, the data provided by the
	OhlohAnalytics tool is already validated and cleansed to be consistent. \\

	Data frames & Daily aggregated to weekly & Monthly & No aggregation of the raw
	data was done in the replication, because the data provided from Ohloh and
	OhlohAnalytics is already presented in monthly time frames. \\

	Time series & Days since first commit & Age in months & Similar to the 'Data
	frames' change, the 'days since first commit' is aggregated in 7 day frames in
	the original study. \citeauthor{karus2013} aggregates the other metrics as
	shown in Table \ref{table:karus_series}. In the replication this was not the
	case as the data provided is already in monthly data frames. The aggregation
	was not done because of issues found with aggregation of LOC as discussed in
	section \ref{section:gapless_wavelets}. \\

	Code churn & Sum of LOC added, modified, and removed & Sum of LOC added and
	removed & Ohloh does not track a modified LOC fact. Therefore, the LOC
	modified metric could not be added to the calculation of LOC churn. \\
\hline
\end{tabular}
\end{table}
\begin{table}[H]
\begin{tabular}{p{20mm}p{30mm}p{30mm}p{6cm}}
	\hline
	\bfseries{Analysis}\rm \\
	\hline

	Signals & $20\times27=540$ signals (see Table \ref{table:karus_series}) &
	$1\times250=250$ signals (i.e., LOC) & Only LOC is used because of its simple
	interpretation and intuitive representation. \\
\hline
\end{tabular}
\end{table}

\subsection*{Discussion}
\subsubsection*{Time frames}
The fact that I used monthly time frames as opposed to weekly time frames, was
because of the fact that Ohloh provides no finer grained time frames than
monthly. Furthermore, there was no reason to believe that finer grained time
frames are necessary for detecting patterns that lead to the end of code
evolution for a project.

The 'end of code evolution', or 'project death' is a phenomenon that reveals in
several months. In the definition of a dead project (see section
\ref{def:dead}), the 'end of code evolution' is the moment that the project
died, which in turn is defined as having no code activity for 12 months in a
row. Therefore, patterns leading to the end of code evolution will last for
months. Weekly data points will probably not add significant value in detecting
warning signs.

\subsubsection*{LOC churn}
The 'LOC churn' metric is defined as the sum of lines of code added, modified,
and deleted. The data used in the replication of the study did not include LOC
modified. Therefore, the LOC churn is calculated as the sum of LOC added and
removed. This differs from the original study. However, there was no reason to
think of this as a threat to validity because the way it is computed was
consistent.

Ohloh inspects differentials when counting lines of code added or deleted. This
means that one line modified will be counted as both one line deleted and one
line added. When calculating the lines of code churn, the lines that were
modified are actually counted twice. Still, it won't introduce problems as the
metric of LOC churn is computed consistently across all projects.

Furthermore, in the replication the LOC churn metric is not used as a signal to
be analysed, but merely as a measure to indicate any code activity for
determining whether a project is dead.

\subsubsection*{Signals}
\citeauthor{karus2013} processed 540 signals in total, as opposed to 250
signals in the replication. The quantity of signals is not related to the
quality of the patterns it is able to detect, because the signals are
processed and analysed independently (i.e., not mixed).

The choice of the signal should depend on the relation between the time and
frequency domains. For time and 'lines of code' (LOC), this relation can be
interpreted with some intuition, which eases validation of the results.

\section*{Evaluation}
The results from the replication are comparable to the results of the original
study. Both studies found patterns in software evolution by using wavelet
analysis. Both studies found patterns that may lead to the end of code
evolution.

\paragraph{}
The difference in results between the original and the replication can be found
in the evaluation of the patterns and the evaluation of wavelet analysis as a
tool to find objective warning signs.

Wavelet analysis works best on continuous data series (section
\ref{section:gapless_wavelets}). Furthermore, the ability of wavelet analysis
to find patterns depends on the data at hand (chapter \ref{conclusions}).

Some patterns show strong characteristics of warning signs, such as patterns of
type A that were detected near the end of evolution of dead projects. As a
result of survival analysis these type A patterns suggest to be a sign of
warning for the death of a project.

\begin{comment}
About original study:
- research questions
- participants
- design
- artifacts
- context variables
- summary of results

About replication:
- motivation
- level of interaction with original researchers
- changes to the original experiment

Comparison of results:
- consistent results
- differences in results

Conclude across studies.

\end{comment}