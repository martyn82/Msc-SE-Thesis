% !TEX encoding = UTF-8 Unicode
\documentclass[9pt,a4paper]{article}
\usepackage{hyperref}

\pagenumbering{arabic}
\setcounter{page}{1}

\title{\bf{Project Plan Msc. Software Engineering}}
\date{}

\begin{document}
\maketitle

\section{General information}
\subsection{Title}
Automatic Means of Identifying Evolutionary Events in Software Development

\subsection{Student name}
Martijn Florian Endenburg

\subsection{Host organisation and group}
VNU Vacature Media (De Persgroep Nederland) \\
Department of Technology, Development \\
mt. Lincolnweg 40 \\
1033 SN, Amsterdam \\
The Netherlands

\subsection{Contact person}
Rutger Pannekoek \\
CTO at VNU Vacature Media \\
phone: 020 - 204 25 21 \\
email:
\href{mailto:r.pannekoek@vnuvacaturemedia.nl}{r.pannekoek@vnuvacaturemedia.nl}

\subsection{Summary}
This master’s project will be a replication of the paper “Automatic means of
identifying evolutionary events in software development” by Siim Karus (2013)
\cite{karus2013}.

Karus has done this study on a data set of 27 open source software
repositories. This data set was chosen carefully to be representative to the
distribution of open source software projects that exist at the time of the
study; i.e., the relative amounts of dead, alive, and active projects.

Karus explored the usage of wavelet analysis on the evolution (time series) of
these projects to see if this analysis method can be used to find objective
warning signs in software projects leading to the end of code evolution.

\section{Project details}

\subsection{Problem analysis}
In closed source projects, many evolutionary events are dictated by the
management. These events can be used to identify project scale and progress,
which allows comparing or co-analysing different projects.

Unfortunately, the evolutionary events in open source software projects are
mostly theoretical with little emperical validation. This also means that
different projects can not be compared in the same scale.

In this study, we are proposing means for detecting evolutionary events in open
source software projects to offer comparable data on their evolution
\cite{karus2013}.

\paragraph{}
\bf{Note: }\rm \emph{The previous seems an assumption by Karus as there is no
citation or study referred to that had found that the evolutionary events in closed source
software projects are mostly dictated by management. So, is this true?}

\bf{Note: }\rm \emph{Can or cannot we generalize the findings of this study to
both open source and industrial/closed source software projects?}

\subsection{Research method}
RQ: Can we use wavelet analysis to find objective warning signs in software
projects leading to the end of code evolution? \cite{karus2013}

\paragraph{}
According to Karus, knowing warning signs could help in:
\begin{itemize}
	\item Choosing OSS to implement in business scenario.
	\item Making timely preparations for decommissioning an OSS project.
	\item Choosing development process that best suites the aims of a software
	project for new OSS projects.
\end{itemize}

\paragraph{}
The research question is a design question and based on a theory how the chosen
analysis method could solve the question. An action research will be done by
choosing a representative data set of open source software projects. These open
source software projects will be analysed to find evolutionary events.

The theory is based on the observations that analysis on time series in
economical and social studies have been successfully done using wavelet
analysis. In the evolution of software projects we are also looking at time
series. Therefore, wavelet analysis seems a logical method to analyse software
project evolution.

\subsection{Expected results}

\subsection{Required expertise}
\begin{itemize}
	\item Expertise on discrete wavelet transformation and analysis.
\end{itemize}

\subsection{Timeline}

\subsection{Risks}
\begin{itemize}
	\item Improper understanding of wavelet transformation and analysis.
	\item Insufficient data to construct a data set that is representative to the
			real-world distribution of open source software projects in terms of scale,
			progress, and activity.
\end{itemize}

\section{Literature study}

\subsection{Bibliography}
\begingroup
\renewcommand{\section}[2]{}%
\begin{thebibliography}{99}
	\bibitem[Karus2013]{karus2013} \hfill \\ S. Karus, \emph{Automatic Means of
	Identifying Evolutionary Events in Software Development}, 1063-6773/13 IEEE,
	2013
\end{thebibliography}
\endgroup

\end{document}
