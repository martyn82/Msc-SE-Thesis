\chapter{Conclusions}
\label{conclusions}

\begin{comment}
- Patterns AB are false-negatives; the indicate a warning sign in some cases, in
other cases they don't.

\end{comment}

\subsubsection{\textit{\ref{itm:question_patterns}: \subQuestionOne}}
The patterns detected using wavelet analysis on LOC signals comprise similar
sequences of LOC series within and across projects. The waveforms of these
patterns vary in all kinds of shapes. Type A patterns all show a stagnation in
LOC changes, that is, between very few and no changes relative to the project
size. Whereas types B, and AB patterns vary between linear, super-linear, and
sub-linear growth or decay (see Figure \ref{figure:type_b_pattern}).

\subsubsection{\textit{\ref{itm:question_successfailure}: \subQuestionTwo}}
The events underlying the patterns vary widely and the event cannot directly be
derived from a pattern. A pattern is merely the symptom or result of an event.
Determining the event that caused the pattern requires an in-depth analysis of
the project and the time period comprising the pattern. Moreover, various
patterns may show similarities, but similar patterns do not always refer to the
same event.

\paragraph{}
In general, the ability to detect a pattern as being a group of similar
sequences within or across projects, depends on the similarity of the sequence
between other sequences within the same analysis. Therefore, it is important to
have a data set that is large enough and representative to the world of OSS
projects to be able to detect patterns that can be generalised as being warning
signs.

\paragraph{}
Another aspect that contributes to the success or failure of wavelet analysis
for software evolution is that the input signal should be free of gaps. In
section \ref{section:gapless_wavelets} this is discussed. In the case an input
LOC signal for analysis is not continuous within the start and end boundaries
of the signal, the wavelet analysis will detect patterns that are not
trustworthy. Wavelet analysis may find more false-positive evolutionary events.

\paragraph{}
The classification of the patterns in this study was done to reduce the
number of patterns that may indicate warning signs. The chosen characteristics
for this classification influenced the reliability of finding warning signs.

Setting up stronger characteristics for the classification treats fewer
patterns as possible warning signs. This yields more false-negatives as some
warning signs might be missed.

On the contrary, setting up weaker characteristics for the classification treats
more patterns as possible warning signs which yields more false-positive
results.

\subsubsection{\textit{\ref{itm:question_warningsigns}: \researchQuestion}}
In this study, I have shown that wavelet analysis is able to find patterns that
increase the chances of a project to end as it appears that there is a relation
between a pattern of type A and the death of a project. However, this relation
was not found to be a causal relation.

Furthermore, as pointed out in the previous paragraphs, the patterns that were
found during analysis are subject to the data set as a whole, and subject to
the classification of the patterns, I cannot conclude that wavelet analysis is
able to find \textit{objective} warning signs in OSS projects.
