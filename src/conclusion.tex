\chapter{Conclusions}
\label{conclusions}

\begin{comment}
- Patterns AB are false-negatives; the indicate a warning sign in some cases, in
other cases they don't.

\end{comment}

In this study, the patterns detected using wavelet analysis on LOC signals
comprise similar sequences of LOC series within and across projects. The
waveforms of these patterns vary in all kinds of shapes. The type A patterns
all show a stagnation of LOC changes, that is, very few changes relative to the
project size.

In general, the ability to detect a 'pattern' as being a group of similar
sequences within or across projects, depends on the similarity of the sequence
between other sequences within the same analysis. Therefore, it is important to
have a data set that is large enough and representative to the world of OSS
projects to be able to detect patterns that can be generalised
[\ref{itm:question_patterns}, \ref{itm:question_successfailure}].

\paragraph{}
Another aspect that contributes to the success or failure of wavelet analysis
for software evolution is that the input signal should be free of gaps. In
section \ref{section:gapless_wavelets} this is argued.

In the case an input LOC signal for analysis is not continuous within the start
and end boundaries of the signal, the wavelet analysis will detect patterns
that are not trustworthy. Wavelet analysis may find more false-positive
evolutionary events [\ref{itm:question_successfailure}].

\paragraph{}
In this study, I have shown that wavelet analysis is able to find patterns that
increase the chances of a project to end. As discussed in section
\ref{section:kp_survival}, it appears that there is a relation between a
pattern of type A and the death of a project, however, it is not found to be a
causal relation.

Furthermore, as the patterns found during analysis are subject to the data set
as a whole, I cannot conclude that wavelet analysis is able to find
\textit{objective} warning signs in OSS projects
[\ref{itm:question_warningsigns}].

\section{Threats to validity}
The following aspects were found which may lead to threats to the validity of
the results.

\begin{description}
	\item[Construct validity] \hfill
	
	\begin{description}
		\item[\rm{Missing historical data}] -- In the analysis of project's
			evolution data, only the data provided by the OhlohAnalytics tool
			\cite{ohlohanalytics, bruntink2013} was used.
			The data before the first data point in the set is not taken into account. It
			is possible that certain evolutionary events happened before the first point.
			These events were not detected as they lay beyond reach of this study.

		\item[\rm{Missing most recent data}] -- The data provided for the study has
			data points until June 2013. Therefore, not the most recent data of the
			projects is used.

		\item[\rm{LOC as activity indicator}] -- The use of LOC (defined as
			lines of code + comments + blanks) as a measure of project activity could be
			false. The LOC will not change between two months whenever the amount of code
			deleted is equal to the amount added. In that case, the churn would be twice
			the LOC added/deleted, but the LOC will stay the same.

			When patterns are detected that show a stagnation in LOC, the suggestion
			would be that the project's activity is decreasing. However, it might also
			be the case that by coincidence it seems activity is decreasing, but in
			reality lots of activity has been going on.

		\item[\rm{Data source}] -- Using only one data source (Ohloh) may
			influence the value of the metrics. Ohloh did the analysis of the project's
			source code repositories. Several studies have shown that the data provided
			by Ohloh needs a thorough examination and cleansing before it can be used
			\cite{bruntink2013, ohlohanalytics, bruntink2014}. The data was initially
			validated and cleansed, but to make it consistent. No actual verification on
			the correctness of the data was conducted.
	\end{description}

	\item[Internal validity] \hfill

	\begin{description}
		\item[\rm{Selection bias}] -- The selection criteria for the data was to have
			at least 12 data points. It turned out, no project in the set was younger
			than 14 months. It might be that these younger projects show significance in
			the ability to detect evolutionary events, possibly refute or confirm
			findings.
	\end{description}

	\item[External validity] \hfill

	\begin{description}
		\item[\rm{Replication}] -- It seems hard to precisely replicate the study
			as there are a multitude of possible configurations to be made that may, in
			the end, influence the results drastically. Such as, the definition of
			similarity between sequences (what deviation is allowed, what is the
			difference of coincidence and a reoccurring sequence), and the definition of
			a pattern (how many occurrences, minimum, and maximum length).
			
			Furthermore, there is the interpretation of what a 'warning sign' should
			look like on a pattern level.
	\end{description}
\end{description}

\section{Future work}
A next step in the research on the use of wavelet analysis for detecting
warning signs in software evolution would be to use a larger data set. It would
be interesting to know if the findings of the type A patterns will be
consistent. The whole usable data set for this study contains 5,986 projects
(section \ref{method:data}).

\paragraph{}
The analysis could be done on other signals. In this study, only LOC was used,
but many other signals can be constructed. LOC evolution measures code activity
to a certain extent, but LOC churn could reflect code activity better. To use
LOC churn properly, the LOC modified fact is also needed.

Other useful metrics for finding warning signs could be team size, developer's
churn (the number of developers added and removed from the team), contributor's
code activity, bug reports, and defects per kLOC.