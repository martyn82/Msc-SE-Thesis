\chapter{Analysis and Conclusions}
\label{analysis}

\section{Sequence analysis}
% answer subquestion one: what patterns can be found using wavelet analysis?


\section{Conclusions}
%Many more sequences were found using filtering. This is expected as scaling is
% better than shifting at finding small coefficients.

\section{Threats to validity}
\begin{comment}
* Is the Ohloh database a representation of the world of OSS projects?
* Is LOC as the sum of source lines of code, comments, and blank lines valid?
* The use of LOC as a measure of project evolution. Does it represent
activity/growth/whatever to say something about the project's status?
* A selection criterion for the projects was a continuous series of subsequent
monthly facts. Maybe the full series of evolution data of a project is needed in
order to find objective signs or to be able to compare different projects.
* Is 250 projects enough to detect patterns and generalise to the world of OSS
projects?
* Is monthly aggregated data fine-grained enough?

\end{comment}

\section{Future work}


\begin{comment}
- Analyse results
- Conclude and interpret results
- Answer research questions
- Threats to validity
- Discussion
- Future work
 
This chapter contains the analysis and interpretation of the results. The
research questions are answered as best as possible given the results that were
obtained. The analysis also discussed parts of the questions that were left
unanswered.

An important topic is the validity of the results.
What methods of validation were used?
Could the results be generalized to other cases?
What threats to validity can be identified?

There is room here to discuss the results of related scientific literature here
as well.
How do the results obtained here relate to other work, and what consequences are
there?
Did your approach work better or worse?
Did you learn anything new compared to the already existing body of knowledge?
Finally, what could you say in hindsight on the research approach by followed?
What could have done better?
What lessons have been learned?
What could other researchers use from your experience?

A separate section should be devoted to ‘future work,’ i.e., possible extension
points of your work that you have identified. Other researchers (or yourself)
could use those as a starting point.

Refer to Chapters 3.7 and 4 in this example thesis at Paul’s
homepage\footnote{http://homepages.cwi.nl/~paulk/thesesMasterSoftwareEngineering/2006/ReneWiegers.pdf}.
\end{comment}
