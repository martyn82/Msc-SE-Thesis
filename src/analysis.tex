\chapter{Analysis and Discussion}
\label{analysis}

\section{Patterns}
\label{section:pattern_evaluation}
There are more than 100,000 times more similar sequences found in the
scale/filter coefficients than in the wavelet/shift coefficients. The large
difference in number of sequences found between the two types of coefficients
is due to the fact that the LOC metric is a cumulative metric. The typical
trend of a LOC signal is growth. This makes finding similar sequences using
shift coefficients (i.e., along the time axis) less likely.

\paragraph{}
No patterns were detected in shift coefficients. This can be explained by the
fact that the 16 similar sequences in shift coefficients are not similar within
the same group of sequences.

Additionally, the shift coefficients are incomparable to the filter coefficients
because they were found in a fundamentally different way of signal
transformation. Mixing both types of coefficients would neglect the way the
coefficients were found and invalidate the patterns comprising sequences of
both types of coefficients.

\paragraph{}
The patterns that were detected show strong similarity among signals. The
similarity was demonstrated in Figure \ref{figure:patterns_plots} in section
\ref{section:seqs_patterns}. The figure presents an arbitrary pattern and its
occurrences across different projects. The figure also demonstrates how wavelet
transforms 'smooths out' differences in details by scaling and filtering the
signals.

\begin{comment}
More about:
- A type A pattern and its occurrences
- Compare the pattern occurrence on coefficients to the projection on a signal
- What do the other non-type A patterns tell us?
\end{comment}

\section{Survivability}
\label{section:kp_survival}
The Kaplan-Meier estimation of the survival function of the projects as shown
in Figure \ref{figure:kp_survival} visualise the survival probability of
projects with and without the occurrence of a type A pattern. The figure shows
a p-value of 0.00835. This is the log-rank value representing the comparability
of the two groups. In general, a p-value $>$ 0.05 indicates the two groups are
incomparable (i.e., the two groups do not show significant differences).
Thus, the p-value of 0.00835 indicates comparable groups. Typically, the larger
the groups, the lower the p-value. However, group sizes of 93 projects seems
large enough to estimate survivability of projects having an occurrence of a
type A pattern.

\paragraph{}
The curve has no censored projects, i.e., there are no projects lost to
follow-up. This increases the reliability of the curve because all projects in
the curve are either dead or alive at any point in the curve (i.e., they are
all 'at risk' of dying).

\paragraph{}
The Kaplan-Meier estimation of the survival function of the projects as shown
in Figure \ref{figure:kp_survival} suggests that projects in group G1 -- the
projects having an occurrence of a type A pattern -- die earlier than the
projects in group G0 -- the projects without an occurrence of a type A pattern.

Figure \ref{figure:kp_survival} shows that the projects of G0 have a survival
chance of approximately 93\% after the first year. In the first 12 months, the
survival chances drop to around 86\%. After that, the survival chance remains
stable for projects in G0.

More than half of the projects in G0 lives longer than 84 months (7 years)
against 72 months (6 years) in G1. A quarter of the projects in G0 lives longer
than 144 months (12 years), against 108 months (9 years) in G1. Two projects of
G0 outlives all projects in G1 after 18 years.

\paragraph{}
During the first year, the projects of G1 have 3\% higher survival chance than
those of G0: approximately 96\%. It is after 4 years that the projects of G1
have a lower survival chance than those in G0. Between 3.2 and 3.8 years of
age, projects in both groups show equal survival chances.\\

\noindent
The four projects in G0 that die in their first 14 months of age are shown in
Table \ref{table:deads_g0}.

\begin{table}[H]
\caption{Dead projects in group G0}\label{table:deads_g0}
\centering
\begin{tabular}{rrlrlr}
\hline
	\textbf{Project} & \textbf{Final age} & \multicolumn{4}{l}{\textbf{Patterns
	found}} \\
	\hline
	586805 & 14 & AB & 6 & B & 1 \\
	587198 & 3 & AB & 9 & B & 0 \\
	587204 & 8 & AB & 15 & B & 4 \\
	588411 & 5 & AB & 12 & B & 0 \\
	\hline
	\textbf{Total} & & & \textbf{42} & & \textbf{5} \\
\hline
\end{tabular}
\end{table}

\noindent
Table \ref{table:deads_g0} shows that the four dead projects of group G0 have
no pattern of type A, which is true by the definition of the group. However,
they do all have occurrences of patterns of type AB. This means that there are
patterns detected in these four projects that occur near and last until the end
of code evolution of these projects. Further evaluation shows that the 42 AB
patterns from these four projects all show a stagnation in LOC change. The
difficulty is that the type AB patterns also occur somewhere else in a dead
project's evolution, which makes it hard by current definitions to identify as
possible warning signs.

\begin{comment}
More about:
- How come G1 have higher survival chances in the first year?
- How come G1 and G0 have equal chances between 3.2 and 3.8 years?
- What are the causes of death in G1? Especially the partial extinction in month
120?
- What is time-to-live between diagnosis and death?
- False-positives: patterns detected as warning signs which are not.
- False-negatives: warning signs undetected.
\end{comment}

\begin{comment}
- Analyse results
- Conclude and interpret results
- Answer hypotheses and research questions
- Threats to validity
- Discussion
- Future work
 
This chapter contains the analysis and interpretation of the results. The
research questions are answered as best as possible given the results that were
obtained. The analysis also discussed parts of the questions that were left
unanswered.

An important topic is the validity of the results.
What methods of validation were used?
Could the results be generalized to other cases?
What threats to validity can be identified?

There is room here to discuss the results of related scientific literature here
as well.
How do the results obtained here relate to other work, and what consequences are
there?
Did your approach work better or worse?
Did you learn anything new compared to the already existing body of knowledge?
Finally, what could you say in hindsight on the research approach by followed?
What could have done better?
What lessons have been learned?
What could other researchers use from your experience?

A separate section should be devoted to ‘future work,’ i.e., possible extension
points of your work that you have identified. Other researchers (or yourself)
could use those as a starting point.

Refer to Chapters 3.7 and 4 in this example thesis at Paul’s
homepage\footnote{http://homepages.cwi.nl/~paulk/thesesMasterSoftwareEngineering/2006/ReneWiegers.pdf}.
\end{comment}
