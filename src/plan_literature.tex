\section{Literature study}

\subsection{Summaries and relevance}

\begin{description}
	\item[\cite{karus2013}] \hfill \\
		This is the base paper of the Master's project.
		Karus has explored the usage of wavelet transforms for the analysis of OSS
		projects evolution. Karus was able to detect close to 1,000 similar patterns
		in a subset of 27 OSS projects. In conclusion, the analysis reveals that
		wavelet analysis can be a powerful and objective tool for identifying
		evolutionary events that can be used as estimation basis or management guide
		in software projects.

		During this Master's project I will replicate and extend the work of Karus by
		comparing patterns and events in order to validate wavelet analysis as a tool
		to predict end of software evolution. This will be done using a larger data
		set and a critical validation of the results found in the base paper.

	\item[\cite{raja2012}] \hfill \\
		Raja and Tretter defined and validated a multidimensional measure of OSS
		project survivability: project viability. This measure has three dimensions:
		vigor, resilience, and organisation. For each of these dimensions they defined
		a viability index. The outcome of the study is that they have been able to
		confirm that vigor, resilience, and organisation are different attributes
		related to project survivability.

		This paper is related to the Master's project in terms of determining whether
		an OSS project is active or dying.

	\item[\cite{samoladas2010}] \hfill \\
		Samoladas et al. studied survival analysis techniques for estimating the
		future development of a FLOSS project. They used duration data from
		FLOSS of thousands of projects from different source code repositories. They
		have applied different survival analysis methods to predict the survivability
		of the projects (i.e., their probability of continuation in the future). They
		have used the application domain and number of committers as metrics that
		affect duration.

		This paper is related to this project on the analysis of project survivability
		measures. It can be used to validate the results found by wavelet analysis.
	
	\item[\cite{elbaum}] \hfill \\
		Elbaum and Munson present a methodology that, as they call it, will produce a
		viable fault surrogate. The measure for estimating the impact of code change
		is of interest in this paper. The authors introduced the notion of \emph{code
		churn}. A code churn is the total lines of code added, modified, and removed.
		It is a precise measurement of software development process and product
		outcomes.

		Code churn is an objective measure of software development progress. It
		measures the total work done. Counting the lines of code (LOC) is not enough
		as there could have been a refactoring of existing code that removed 20\% of
		the lines. With counting lines you would miss the work done to remove that
		code. Therefore, code churn is a better way to measure the development
		progress.
	
	\item[\cite{dumas}] \hfill \\
		Karus and Dumas studied a means to estimate future code maintenance effort in
		order to plan software maintenance activities. In this paper they used machine
		learning techniques to unveil predictors of yearly cumulative code churn of
		software projects on the basis of metrics extracted from version control
		systems. They have shown that a code churn estimation model built purely with
		organisational metrics is superior to one built purely with code metrics.
		However, a combined model provides the highest predictive power. Code metrics
		in general are complementary to organisational metrics for the purpose of
		estimating code churn.

		The paper is related to the project in ways of determining future code churn.
		It will be of use to predict software evolution events.
\end{description}

\subsection{Bibliography}
\begingroup
\renewcommand{\section}[2]{}%
\begin{thebibliography}{99}
	\bibitem[1]{karus2013} S. Karus, \emph{Automatic Means of
	Identifying Evolutionary Events in Software Development}, 1063-6773/13 IEEE,
	2013

	\bibitem[2]{raja2012} U. Raja and M.J.
	Tretter, \emph{Defining and Evaluating a Measure of Open Source Project Survivability},
	0098-5589/12 IEEE Transactions on Software Engineering vol.38 no.1, jan-feb
	2012

	\bibitem[3]{samoladas2010} I. Samoladas, L. Angelis, I.
	Stamelos, \emph{Survival Analysis on the Duration of Open Source Projects},
	Information and Software Technology 52, 2012, p.902-922

	\bibitem[4]{elbaum} S.G. Elbaum and J.C. Munson, \emph{Code Churn: A Measure
	for Estimating the Impact of Code Change}, IEEE Software Maintenance, 1998

	\bibitem[5]{dumas} S. Karus and M. Dumas, \emph{Code Churn Estimation Using
	Organisational and Code Metrics: An Experimental Comparison}, Institute of
	Computer Science Journal vol. 54, issue 2, 2012

\end{thebibliography}
\endgroup
