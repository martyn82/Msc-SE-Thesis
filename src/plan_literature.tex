\section{Literature study}

\subsection{Summaries and relevance}

\begin{description}
	\item[\cite{karus2013}] \hfill \\
		This is the base paper of the Master's project.
		Karus has explored the usage of wavelet transforms for the analysis of OSS
		projects evolution. Karus was able to detect close to 1,000 similar patterns
		in a subset of 27 OSS projects. In conclusion, the analysis reveals that
		wavelet analysis can be a powerful and objective tool for identifying
		evolutionary events that can be used as estimation basis or management guide
		in software projects.
		
		This Master's project will extend the work of Karus by comparing
		patterns and events in order to predict end of software evolution.

	\item[\cite{raja2012}] \hfill \\
		Raja and Tretter defined and validated a multidimensional measure of OSS
		project survivability: project viability. This measure has three dimensions:
		vigor, resilience, and organisation. For each of these dimensions they defined
		a viability index. The outcome of the study is that they have been able to
		confirm that vigor, resilience, and organisation are different attributes
		related to project survivability.
		
		This paper is related to the Master's project in terms of determining whether
		an OSS project is active or not, the Master's project provides more. The study
		of identifying events that predict a project's viability might be more
		valuable because they do not only confirm that there are attributes related
		to viability, but also about the current and future condition of a project.

	\item[\cite{samoladas2010}] \hfill \\
		Samoladas et al. studied survival analysis techniques for estimating the
		future development of a FLOSS project. They used duration data from FLOSS
		thousands of projects from different source code repositories. They have
		applied different survival analysis methods to predict the survivability of
		the projects (i.e., their probability of continuation in the future). They
		have used the application domain and number of committers as metrics that
		affect duration.
		
		I find that using metrics such as number of committers (number of developers)
		is a good way to measure project popularity, but application domain may not be
		the most objective way of measuring project survivability. Therefore, I
		believe that analysing time series is a more objective approach for measuring
		project viability.
	
	\item[\cite{elbaum}] \hfill \\
		Elbaum and Munson present a methodology that, as they call it, will produce a
		viable fault surrogate. The measure for estimating the impact of code change
		is of interest in this paper. The authors introduced the notion of \emph{code
		churn}. A code churn is the total lines of code added, modified, and removed.
		It is a precise measurement of software development process and product
		outcomes.
		
		Karus believes, and I agree, that code churn is an objective measure of
		software development progress. It measures the total work done. Counting the
		lines of code (LOC) is not enough as there could have been a refactoring of
		existing code that removed 20\% of the lines. With counting lines you would
		miss the work done to remove that code. Therefore, code churn is a better
		way to measure the development progress.
	
	\item[\cite{dumas}] \hfill \\
		Karus and Dumas studied a means to estimate future code maintenance effort in
		order to plan software maintenance activities. In this paper they used machine
		learning techniques to unveil predictors of yearly cumulative code churn of
		software projects on the basis of metrics extracted from version control
		systems. They have shown that a code churn estimation model built purely with
		organisational metrics is superior to one built purely with code metrics.
		However, a combined model provides the highest predictive power. Code metrics
		in general are complementary to organisational metrics for the purpose of
		estimating code churn.
		
		In this paper, Karus and Dumas have shown that it is possible to use code
		churn combined with machine learning to estimate future code churn. It has
		been done by Karus in the base paper of this Master's project. This is useful
		if we want to be able to predict software evolution.
\end{description}

\subsection{Bibliography}
\begingroup
\renewcommand{\section}[2]{}%
\begin{thebibliography}{99}
	\bibitem[1]{karus2013} S. Karus, \emph{Automatic Means of
	Identifying Evolutionary Events in Software Development}, 1063-6773/13 IEEE,
	2013

	\bibitem[2]{raja2012} U. Raja and M.J.
	Tretter, \emph{Defining and Evaluating a Measure of Open Source Project Survivability},
	0098-5589/12 IEEE Transactions on Software Engineering vol.38 no.1, jan-feb
	2012

	\bibitem[3]{samoladas2010} I. Samoladas, L. Angelis, I.
	Stamelos, \emph{Survival Analysis on the Duration of Open Source Projects},
	Information and Software Technology 52, 2012, p.902-922

	\bibitem[4]{elbaum} S.G. Elbaum and J.C. Munson, \emph{Code Churn: A Measure
	for Estimating the Impact of Code Change}, IEEE Software Maintenance, 1998

	\bibitem[5]{dumas} S. Karus and M. Dumas, \emph{Code Churn Estimation Using
	Organisational and Code Metrics: An Experimental Comparison}, Institute of
	Computer Science Journal vol. 54, issue 2, 2012

\end{thebibliography}
\endgroup
