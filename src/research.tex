\chapter{Research}
\label{research}

\section{Data}
Data was gathered from Ohloh.net\footnote{http://www.ohloh.net}. Ohloh is a
project by the initiative of Black Duck Software. The Ohloh universe contains
approximately 600,000 open-source software projects in January 2014. The
evolution data of all these projects is tracked and monthly aggregated.

\subsection{Data gathering}
For the data gathering and validation we used the tool \emph{OhlohAnalytics }\rm
created by Magiel Bruntink \cite{ohlohanalytics}. This tool was created for the
replication of the research ``\emph{An Initial Quality Analysis of the Ohloh
Software Evolution Data}\rm'' by M. Bruntink. This tool provides us with an
initial data set of more than 10,000 OSS projects gathered in July 2013.\\

\noindent
The following steps are done in the data gathering process:
\begin{enumerate}
	\item Data collection.
	\item Validation and cleaning.
\end{enumerate}

\subsubsection{Data collection}
During the data collection step the names of the projects tracked by Ohloh are
gathered. Then for each project, monthly facts (so called \emph{factoids}\rm)
are gathered.
These facts include size facts (lines of code (LOC), blank lines, commented lines), activity facts
(number of commits, number of contributors, LOC added/removed), enlistments
(i.e., source code repositories for the project), language (main programming
language, other languages).

\subsubsection{Validation and cleaning}
The initial validation step does sanity checks and filtering on the collected
data. It filters out the projects that have a wrongly configured repository,
has missing data, negative LOC added, negative LOC removed, negative LOC,
negative number of commits, negative number of contributors, or have
inconsistent values where the difference between the previous month and current
month numbers are inexplainable.

\subsection{Data selection}
From the 10,000 projects gathered by the OhlohAnalytics tool, a sample of 250
projects have been selected. This sample must be a representative selection of
the projects tracked by Ohloh. For this selection process a script called
\emph{SampleSoftwareProjects }\rm was used and provided with the OhlohAnalytics
tool. The SampleSoftwareProjects tool was developed by Nagappan et al at
Microsoft Research \cite{nagappan}.

The script takes a master data set containing interesting facts about the
projects tracked by Ohloh. These facts include activity facts (user count,
average rating, last updated), project age, size facts (LOC), and language
facts. The projects are scored by each of these dimensions.\\

Taking one sample project as base line the script evaluates projects from the
master set and adds them to the sample if it increases the overall coverage to
be close to that of the master set as a whole. The resulting data set is aimed
to be 100\% representative.\\

\noindent
For this study we took the project \emph{Mozilla Firefox }\rm as basis for
the sample. The resulting data set of 250 projects has a representativeness of
100\% according to the SampleSoftwareProjects script.

\subsection{Data validation}
As a final step we have validated the 250 projects for data completeness. This
is not a trivial step as we want to be able to analyze the project's evolution
data. Therefore, the more complete this data is, the more likely we will be able
to detect patterns. For all 250 projects we counted the number of monthly
factoids before and after cleansing and computed the percentage of evolution
data coverage per project. For the least covered project we have 5.1\% of its
evolution data, the highest coverage is 99.7\%. The average coverage is 78.6\%.
19 projects are covered for 33\% or less. 156 projects are covered for more than
80\%.

\section{Metrics}
The following metrics are used and measured monthly:

\begin{description}
	\item[Age in days] \hfill \\
		This is the number of days since the first factoid of a project. The factoids
		are monthly statistics, thus the age in days is determined by the date of the
		factoids. We have taken a constant of 22 working days per month. The average
		month has 30.4 days, divided by 7 gives 4.34 weeks, multiplied by 5 working
		days gives 21.7 working days per month.

	\item[Active developers] \hfill \\
		At Ohloh this is called the number of contributors and is the actual number of
		developers that have made at least one commit in the month of the factoid.
	
	\item[Lines of code (LOC)] \hfill \\
		For this study we use the same definition of LOC as in the original study by
		Karus; blank lines and commented lines are also included in the LOC counts.
		The reason for this is that we want to be able to measure activity and a
		comment added or removed is also considered project activity related to code.

	\item[Code Churn] \hfill \\
		Is the sum of LOC added, LOC removed, and LOC modified. The modified LOC is
		not tracked at Ohloh but is aggregated in LOC added and LOC removed. A
		modified LOC is basically a line that is both removed and added.

\end{description}

\noindent
For time series dimensions we used \emph{Age in days }\rm and \emph{Active
developers}\rm. For signals we used \emph{Active developers}\rm, \emph{LOC}\rm,
and \emph{Code churn}. Table \ref{tab:series} shows what time series are
plotted against what signals.

\begin{table}
\centering
	\caption{Time series and signals}
	\begin{tabular}{| p{4cm} | p{4cm} | p{4cm} |}
	\hline
	Time series & Signal & Description \\ \hline
	Age in days & Active developers & Team size over time. \\
				& Lines of code & LOC size over time. \\
				& Code churn & LOC change over time. \\ \hline
	Active developers & Code churn & LOC change per team size \\ \hline
	\end{tabular}
\label{tab:series}
\end{table}

\section{Analysis}
The analysis of the data is done per project, per time series, per signal
series. For 250 projects and 4 two-dimensional series this means that were a
total of 1,000 analysis runs.\\
The analysis was conducted in the following steps:
\begin{enumerate}
	\item Data aggregation.
	\item Wavelet transform.
	\item Similar sequence identification.
	\item Similar sequence grouping.
\end{enumerate}

\subsection{Data aggregation}
The first step of the analysis is aggregregation of the data by the time series
under analysis. For instance, if there exists multiple factoids for the same
time series value, this data has to be aggregated. Depending on the signal
variable the aggregation was a summation, mean, max, or min value. The output of
the aggregation step a set of data for one project consisting of one data point
per time value.

\subsection{Wavelet transform}
The aggregated data is fed to the discrete wavelet transformation. This step is
divided into two ways of wavelet transformation each using a different
coefficient. One wavelet transform is done using the filter coefficent, the
second using the scaling coefficient. The results of the wavelet transform step
are sequences of project evolution data which are patterns of interest. These
sequences have characteristics such as length and amplitude that are beyond a
predefined threshold.

\subsection{Similar sequence identification}
The results of the wavelet transforms for each dimension are analyzed and
compared to find similar sequences in different projects.

\subsection{Similar sequence grouping}

\begin{comment}
This chapter reports on the execution of the research method as described in Chapter 3.

If the research has been divided into phases (e.g., using sub questions) the
phases are introduced, reported on and concluded individually. If needed this
Chapter could be split up to balance out the sizes of all Chapters.
An example Research Chapter is provided as Chapter 3 at Paul’s home
page\footnote{http://homepages.cwi.nl/~paulk/thesesMasterSoftwareEngineering/2006/ReneWiegers.pdf}.
\end{comment}
