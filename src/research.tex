\chapter{Research}
\label{research}

This chapter describes the execution of the research.

\section{Data set}
The data set of 250 projects was selected using the method described in section
\ref{method:data}. It was conducted in the steps described in the following
paragraphs.

\subsection{Initial analysis, validation, and cleansing}
The evolution data of a project is available as monthly data from Ohloh.net. The
raw data of 12,360 projects was collected from Ohloh.net in July 2013 by
\citet{bruntink2014} as part of his research. It was then analysed, validated,
and cleansed accordingly by the tool \emph{OhlohAnalytics }\rm
by \citet{ohlohanalytics}.

This step consists of the analysis, validation, and cleansing of the data by
detecting inconsistent values and removing these records from the data set.
Additional fields are derived from and added to the raw data for convenience.

The complete list of data fields are shown in table \ref{table:fields}. The
column 'Source' specifies if the field is either 'Raw' data (i.e., directly
available from Ohloh), or if it is added by the OhlohAnalytics tool as derived
from an operation on one or more other fields. In the latter case, the fields
are listed. The 'lines of code' (LOC) metric is equal to the LOC metric used by
Ohloh. This is the total number of source lines of code at the given time of
analysis. These lines include all lines of source text for the project in any
programming language, but excludes blank lines and comments.

The result of this step is a consistent data set of evolution data of 10,811
projects.

\begin{table}
	\caption{Monthly data fields}
	\begin{tabular}{p{4cm} p{3cm} p{7.5cm}}
		\hline
		\bfseries{Field}\rm & \bfseries{Source}\rm &
		\bfseries{Description}\rm \\ \hline
		
		project\_name\_fact & Raw & The name of the project at Ohloh. \\
		\hline
		
		\bfseries{Activity}\rm \\ \hline

		abs\_loc\_growth & $loc\_added\_fact$, $loc\_deleted\_fact$ & The number of
		'lines of code' (LOC) that the project has grown (or shrank) this month. \\

		blanks\_added\_fact & Raw & The number of blank lines added to source text
		this month. \\

		blanks\_deleted\_fact & Raw & The number of blank lines deleted from source
		text this month. \\

		comments\_added\_fact & Raw & The number of lines of comments added in source
		text this month. \\

		comments\_deleted\_fact & Raw & The number of lines deleted from source text
		which are comments; for this month. \\

		commits\_fact & Raw & The total number of commits made this month. \\

		contributors\_fact & Raw & The total number of contributors who made at least
		one commit this month. \\

		ind\_loc\_growth & $loc\_fact$, $abs\_loc\_growth$ & The relative growth of
		the project measured in LOC for this month. \\

		loc\_added\_fact & Raw & The number of LOC added this month. \\

		loc\_deleted\_fact & Raw & The number of LOC deleted this month. \\
		\hline
		
		\bfseries{Analysis}\rm \\ \hline

		age\_in\_months & $month\_fact$, $year\_fact$ & The age of the project in
		months measured since the first data point; starts at 0. \\

		age\_in\_years & $age\_in\_months$ & The age of the project in years measured
		since the first data point; starts at 0. \\

		cumulative\_commits\_fact & $commits\_fact$ & The total number of commits
		since the first data point (i.e., where $age\_in\_months = 0$). \\

		main\_language\_fact & Raw & The programming language having the highest LOC
		value for the project in this month (XML and HTML are ignored). \\

		month\_fact & Raw & The month value of this month's analysis. Extracted from
		$udpated\_at$ field. \\

		year\_fact & Raw & The year value of this month's analysis. Extracted from
		$updated\_at$ field. \\
		\hline
		
		\bfseries{Size}\rm \\ \hline

		blanks\_fact & Raw & The total number of blank lines in source text this
		month. \\

		comment\_ratio\_fact & Raw & The fraction of net lines in source text which
		are comments; for this month. \\

		comments\_fact & Raw & The total number of lines in source text which are
		comments; for this month. \\

		loc\_fact & Raw & The total number of LOC this month. \\

		\hline
	\end{tabular}
\label{table:fields}
\end{table}

\subsection{Subsequent data series}
The evolution data of the 10,811 projects contained gaps. We expect a
subsequent series of data is necessary to be able to analyse time series for a
project. Therefore, a number representing the fraction of continuity of the
data was needed. For each project the difference between the minimum and
maximum values of the $age\_in\_months$ fact is taken, added by one, giving us
the expected number of data points for a project. The calculation of the
fraction of total evolution data is done for each project.

After the fractions of the total evolution data for each project are caculated,
we can filter these projects and keep only the projects that have all data points
between minimum and maximum $age\_in\_months$. From the set of 10,811 projects,
a total number of 6,418 projects is left.

\subsection{Minimal sequence length}
As a final step in validating and cleaning the master data set, we will be
keeping only the projects that have at least 12 data points (i.e., have a
subsequent evolution period of at least 1 year).

After this selection, a set of 5,986 projects is left.

\subsection{Sample selection}
For the selection of a sample of 250 projects, we use the tool created by
\citet{nagappan}. This tool takes a, possibly atomic, sample set and selects
additional projects to add to the sample that increase the overall
representativeness of that sample.

The tool iteratively selects 250 projects and adds it to the sample. Each
additional project is selected by its score to maximally increase the
representativeness of the sample as a whole, compared to the master data.

The master data is a list of 20,028 projects tracked by Ohloh.net delivered with
this tool. A pre-filtering of this master data was done to make the tool select
only the projects that appear in our data set of 5,986 projects. After
pre-filtering, a subset of the master data of 1,588 projects is left. From this
subset, a sample of 250 projects was selected.

The resulting sample of 250 projects was scored against the initial master data
of 20,028 projects and scores a 99.5\% representativeness.

\paragraph{}
The evolution data of 250 distinct projects make a total of 22,943 data points.
The biggest project having 321 data points, the smallest having 14 data points.

\section{Analysis}

The 250 projects are analysed by selecting time series (signals) of their
evolution data. In this research, analysis is done on modeling $age\_in\_months$
on the time domain, and $loc\_fact$ on the frequency domain. The signals per
project represent the value of the 'lines of code' at a given 'age in months'.

\subsection{Wavelet transform and analysis}

For the wavelet transform and analysis, we have used R Statistics Suite
with packages ``wavelets'', ``chron'' and ``zoo''. The Wavelets R package
contains an implementation of discrete wavelet transform functions and the Haar
filter. The Haar filter is used by \citet{karus2013} in his research because of
its simplicity and ease of interpretation. As this study is a replication of
\citeauthor{karus2013}' study, the choice of using the Haar filter was made.

The R scripts used for these three steps are based on the scripts created and
used by \citeauthor{karus2013} in the initial research. Small adjustments have
been made to make them compatible with our data set.

\subsubsection{Wavelet transform}
Each project's signal is separately transformed using discrete wavelet
transform. The results of the transforms are saved per project. This is the raw
data after decomposing signals and is saved for further analysis.

\subsubsection{Similar sequence identification}
The similar sequence identification found 1,669,448 sequences that occurred
at least 2 times in the data. Only 16 sequences were found using wavelet
coefficients, the other 1,669,432 sequences were found using filter
coefficients. This is expected as filter coefficients are better than wavelet
coefficients at finding small changes in signals.

\subsubsection{Similar sequence grouping}
The last step in the analysis of the found sequences aggregates and groups the
sequences together and keeps only the 'popular' sequences. As mentioned, a
sequence is considered popular if it appears at least 3 times in the data.

The results of this step revealed a total of 16,049 of such popular sequences.
All the popular sequences were found in the sequences by filter coefficients,
and none in the sequences by wavelet coefficients. The sequences occurred at
least 4 times, and at most 1,511 times. They occur in at least 1 and at most 204
projects. Their lengths differ between 4 points and 19 points across various
decomposition levels (levels of detail).

\section{Pattern identification}



\begin{comment}
This chapter reports on the execution of the research method as described in Chapter 3.

If the research has been divided into phases (e.g., using sub questions) the
phases are introduced, reported on and concluded individually. If needed this
Chapter could be split up to balance out the sizes of all Chapters.
An example Research Chapter is provided as Chapter 3 at Paul’s home
page\footnote{http://homepages.cwi.nl/~paulk/thesesMasterSoftwareEngineering/2006/ReneWiegers.pdf}.
\end{comment}
