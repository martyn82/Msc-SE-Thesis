\section{Problem analysis}
\paragraph{}
In closed source software projects, many evolutionary events are dictated by the
management. These events can be used to identify project scale and progress,
which allows comparing or co-analysing different projects.

Unfortunately, the evolutionary events in OSS projects are mostly theoretical
with little empirical validation. This also means that different projects can
not be compared in the same scale.

In this study, we are proposing means for detecting evolutionary events in OSS
projects to offer comparable data on their evolution \cite{karus2013}.

\paragraph{}
At VNU Vacature Media developers extensively use OSS projects to support their
software projects. These OSS projects include Apache web server, Zend Framework,
and MySQL database server.

Amongst the previous well-known OSS projects, less known OSS projects are used.
Therefore, it is valuable to be able to detect warning signs in the
evolution of such projects. The detection and monitoring of their evolution is
useful to decide whether to choose for and/or continue to use an OSS project.

\paragraph{}
The survivability of OSS projects has been of interest by many researchers. In
an industrial environment, a project is considered successful if it is completed
within time and budget constraints. In OSS projects, these constraints do not
always exist. The indicators of success for OSS projects differ from the one
study to the other. Most studies used measures as number of developers,
developer's satisfaction, bug resolution time, level of activity
\cite{samoladas2010}.

\paragraph{}
There has been a study by Raja et al. \cite{raja2012} on defining a measure of
OSS project survivability. They have been looking for vitality of OSS projects.
The question whether an OSS project is active or not is not of our specific
interest. We want to be able to predict the OSS project evolution.

\paragraph{}
Karus used wavelet analysis on time series in software evolution to be able to
detect events leading to the end of code evolution. We dive deeper into these
events to be able to find warning signs early in the evolution of OSS projects.
More on this method in the next section.
