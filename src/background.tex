\chapter{Background}
\label{background}

The survivability of OSS projects has been of interest by many researchers. In
an industrial environment, a project is considered successful if it is completed
within time and budget constraints. In OSS projects, these constraints do not
always exist. The indicators of success for OSS projects differ from the one
study to the other.

\paragraph{}
In a study by Samoladas et al. a method for survival analysis on OSS projects
was proposed. These methods were employed to predict the survivability of the
projects by examining their duration, combined with other characterstics such
as application domain and number of committers. Although these metrics give
insight in the survivability chances of a project, it was also found that
adding a developer to the team of contributors increased the survivability of
the project substantially. This relation cannot be linear; there must be a
point where adding a developer to the project has no (substantial) increase of
survivability.\\
The authors of the paper proposed two main research issues to be addressed in
the future. The first one is to add more projects to the study with possibly a
different categorization.
And second, the effects of more project parameters, such as programming
language should be examined. This is not trivial since typically more than one
language is used in each project \cite{samoladas2010}.

\paragraph{}
A study by Raja and Tretter on defining a measure of OSS project survivability.
They have been looking for vitality of OSS projects: the ability of a project
to grow and maintain its structure in the presence of perturbations. They
identified three dimensions of project viability: vigor -- the ability of a
project to grow --, resilience -- the ability of a project to recover from
disturbances --, and organization -- the structure exhibited in the project.
These dimensions represent three distinct characteristics of project viability
\cite{raja2012}. This measure is of use to determine survivability of a
project, however, it is a snapshot of a point in time. It does not take into
account the events prior to this point in time. Nor does it enable prediction of
survivability in the near future. Therefore, it can be challenging to get an
objective view on a project's survivability as a representative point in
time has to be selected.

\paragraph{}
Crowston et al. identified measures that can be applied to assess the success of
OSS projects. The authors used the DeLone and McLean model of information
systems success to evaluate OSS project success \cite{delone1992}. The aspects
identified by DeLone and McLean are elaborated; output of systems development -- it is
believed that a project that has a high frequency of releases is healthy --,
process of systems development -- the number of developers, the individual
level of activity, and cycle time (time between releases) --, and
project effects -- employment opportunities of the contributors, individual
reputation, and knowledge creation. In this study it was found that many of
these aspects are indicators of OSS project success \cite{crowston2003}.
Although the cycle time is an aspect that could be measured automatically, the
other aspects are \emph{soft}\rm. Soft in the way that it is hard to get into
numbers. This makes automatic evaluation and analysis on these aspects very
hard.

\paragraph{}
Another study conducted by Crowston et al. extends the previous study by using
Free/Libre Open-Source Software (FLOSS) projects. They found that the number of
developers as a simple count of developers is a somewhat flawed number as
it aggregates the number of developers leaving and the number of developers
joining a project. A 'churn' of the developers or a 'tenure' of individuals
would be more appropriate.
This study took projects from one source code repository,
SourceForge, instead of different repositories \cite{crowston2006}.

\paragraph{}
A study conducted by Wang has shown that warning signs can be found in
six crucial factors of OSS projects success: developer participation effort,
developer service quality, software license restrictiveness, targeted users,
community social network ties, and community quality of social ties
\cite{wang2012}.
These factors play a role in OSS project success, however, these factors are not
easily analyzed automatically.


\begin{comment}
This chapter contains all the information needed to put the thesis into
context. It is common to use (a revised version) of your literature survey for
this purpose.
It is important to refer from your text to sources you have used, as listed in
your bibliography section (appendix). For example, “XP is a recent agile
development method [1]” is a common style of doing this, where the following
entry would be included in your bibliography:
[1] K. Beck, E. Gamma, Test infected: Programmers love writing tests, Java
Report 3 (7) (1998) 51–56.
If you want to refer to books you have read as part of the curriculum, you can
also do so in this way.
Have a look at Chapter 2 of this example thesis at Paul’s
homepage\footnote{http://homepages.cwi.nl/~paulk/thesesMasterSoftwareEngineering/2006/RichardKettelerij.pdf}.
\end{comment}