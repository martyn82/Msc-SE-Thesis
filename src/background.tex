\chapter{Background}
\label{background}

This chapter provides background information by contributing to a basic
understanding of the field of software evolution, the assessment of
survivability and success of OSS projects, and wavelet analysis.

\section{Software evolution}
The analogy of the term \emph{evolution }\rm in the field of software
engineering was first used by Lehman in his laws of software evolution
\cite{lehman}. Software does not evolve by intrinsic feedback loops like
evolution in plants and animals, but by extrinsic feedback that comes from the
operational domain.

\paragraph{}
Lehman identified eight laws that are the driving factors of changes in software
systems. These laws can be roughly categorised into two categories: laws
related to the product (e.g., source code, and internal quality), and laws
related to the process (e.g., organisation, development process, business
requirements and user satisfaction).

\paragraph{}
The laws related to the software process tell us that in order for a software
system to stay useful and satisfactory, it needs to keep changing according to
users' needs. What needs to change is fed through the feedback system.

On average and in terms of activity, the organisation of a software development
process does not change much during a product's life time (law IV:
Conservation of Organisational Stability).

\paragraph{}
The laws related to the software product tell us that as the source code of the
software system changes, its complexity will increase and its quality will
decline unless proper actions are taken to maintain it or reduce the complexity
and improve its quality. However, the contents of the product will be
statistically invariant during the active life time of a product.

\paragraph{}
Lehman's laws of software evolution are valid for both closed source software
and open-source software. These laws form the foundation of what we understand
as software evolution.

\paragraph{}
The evolution of a software system comprises various measures on different
moments in time. For instance, the lines of code metric over a project's life
time is a way to measure the evolution of lines of code in a software project.



\section{Project survivability}
%% perhaps restructure in Project success, and Project survivability %%
In a study by Samoladas et al. a method for survival analysis on OSS projects
was proposed \cite{samoladas2010}. The authors used duration data regarding
Free/Libre Open-Source Software (FLOSS) projects to predict the survivability of
the projects by examining their duration, combined with other characterstics
such as application domain and number of committers. These metrics give insight
in the survivability chances of a project. It was also found that adding a
developer to the team of contributors increased the survivability of the
project substantially.

The authors proposed two main research issues to be addressed in the future. The
first one is to add more projects to the study with possibly a different
categorisation. And second, the effects of more project parameters, such as
programming language should be examined. This is not trivial since typically
more than one language is used in each project.

\paragraph{}
A study by Raja and Tretter on defining a measure of OSS project survivability
\cite{raja2012}.
They have been looking for vitality of OSS projects: the ability of a project
to grow and maintain its structure in the presence of perturbations. They
identified three dimensions of project viability: vigor -- the ability of a
project to grow --, resilience -- the ability of a project to recover from
disturbances --, and organisation -- the structure exhibited in the project.
These dimensions represent three distinct characteristics of project viability.

This measure is of use to determine survivability of a project, however, it is
a snapshot of a single point in time. It does not take into account the events
prior to this point in time, nor does it enable prediction of survivability in
the near future. Therefore, it can be challenging to get an objective view on a
project's survivability as a representative point in time has to be selected.

\paragraph{}
Crowston et al. identified measures that can be applied to assess the success of
OSS projects \cite{crowston2003}. The authors used the DeLone and McLean Model
of Information Systems Success to evaluate OSS project success
\cite{delone1992}. The aspects identified by DeLone and McLean are elaborated;
output of systems development -- it is believed that a project that has a high
frequency of releases is healthy --, process of systems development -- the
number of developers, the individual level of activity, and cycle time (time
between releases) --, and project effects -- employment opportunities of the
contributors, individual reputation, and knowledge creation. In this study it
was found that many of these aspects are indicators of OSS project success.

Although the cycle time is an aspect that could be measured automatically, the
other aspects such as employment opportunities, reputation, and knowledge
creation are very hard to get into numbers and therefore hard to automate.

\paragraph{}
Another study conducted by Crowston et al. extends the previous study by using
Free/Libre Open-Source Software (FLOSS) projects \cite{crowston2006}. In
addition to what was found in the previous study, they had found that the number
of developers as a simple count of developers is a flawed number as it
aggregates the number of developers leaving and the number of developers
joining a project. A 'churn' of the developers or a 'tenure' of individuals
would be more appropriate.

\paragraph{}
A study conducted by J. Wang has shown that warning signs can be found in
six crucial factors of OSS projects success: developer participation effort,
developer service quality, software license restrictiveness, targeted users,
community social network ties, and community quality of social ties
\cite{wang2012}.

\paragraph{}
Karus explored a method known as \emph{wavelet analysis }\rm to analyse software
evolution data \cite{karus2013}. The wavelet analysis interprets evolution data
as a series of signals and is able to find sequences in this signal. The
sequences of multiple projects can be compared in order to find recurring
patterns. Karus was able to detect 998 similar patterns across 27 OSS projects.
He concluded that wavelet analysis can be a powerful tool for identifying
evolutionary events.





\section{Wavelet analysis}
A wavelet is a portion of a continuous-time signal (also known as waveform).
Opposed to the continuous wave which has infinite duration, the wavelet has a
finite duration.

The analysis of waves is being used in many fields, such as, economics,
seismology, (astro-)physics, and computer science.

\subsection{Waveforms}
A waveform, or wave, is a signal (time-series) in a two-dimensional space.
Waveforms are used to model many types of signals, such as audio,
electromagnetic (light), gravitational, and quantum mechanical waves.

\paragraph{}
In many models of waveforms the two dimensions typically represent the time and
frequency domains. An audio signal is an example of a waveform modeled that way.

In visualisations of waveforms, the time domain is often plotted on the
horizontal axis, and the frequency domain on the vertical axis. In the example
of an audio wave, the frequency domain may represent the amplitude, or the
frequency, and the time the duration of the amplitude or frequency value.

However, the time domain is not specifically bound to time intervals only. A
less intuitive, but realistic approach would be to model amplitude in the time
domain. That way the frequency relation to amplitude of a signal can be
analysed, but the time information is lost.

\subsection{Discrete wavelet transformation}
To be able to analyse and compare different wavelets, we need a way to scale
the signal. Discrete wavelet transformation is the operation of applying a
filter function, or set of filter functions (i.e., filter bank) to the wavelet.
This is a way of sampling the signal at different intervals giving a natural
means of scaling the signal \cite{karus2013}.

\paragraph{}
Wavelet analysis is similar to Fourier analysis with the difference that
wavelets deal with time and frequency information, and Fourier transform deals
with frequency information only. Wavelet analysis is the analysis of signals
(time-series) by decomposing the signal into wavelet/shift\footnote{Shifting,
or translating, is the operation of moving the wavelet in the time domain.}
coefficients and scaling/filter\footnote{Filtering, or dilating, is the
operation of scaling the wavelet in the frequency domain.} coefficients based
on wavelet functions (e.g., filters) \cite{karus2013}. The decomposition can be
repeated on the scaling/filter coefficients until the number of resulting
wavelet/shift coefficients is smaller than the filter length. The filter length
is a property of the filter function being used.

\paragraph{}
Wavelet transform has proven important in signal processing thanks to its
inherent properties which allow comparisons at different scales and shifts.
Compared to other time series analysis techniques, the main advantages of
wavelet transformations in the analysis of signals are \cite{karus2013}:
\begin{itemize}
	\item Scaling coefficients allow fuzzy matching as differences in details are
	'smoothed out'.
	\item Filter coefficients allow detection of small anomalies in series.
	\item Transform levels make series of different lengths or scale comparable.
\end{itemize}

\subsection{Wavelet functions}
A wavelet function is a function that defines a wavelet. In general, a wavelet
function is any operation on an existing wavelet \cite{wadkar}. Many wavelet
functions exist and differ largely in complexity and applicability depending on
the signal of interest.

A wavelet can be defined in the following ways, given that $T$ is the set of
time values of the signal, and $F$ the set of frequency values of the signal.
\begin{description}
	\item[Wavelet function (mother wavelet)] \hfill \\ $\Psi: T \longrightarrow
	F$\\ $\Psi(t) = f$, such that $t \in T$ and $f \in F$.\\
	This function maps $T$ onto $F$ and thus produces the shape of the wavelet.

	\item[Scaling function (father wavelet)] \hfill \\ $\Phi: F \longrightarrow
	T$\\ $\Phi(f) = t$, such that $t \in T$ and $f \in F$.\\
	This function maps $F$ onto $T$ and thus produces the scale of the wavelet.

	\item[Scaling filter] \hfill \\ A low-pass filter of length $2N$ and sum 1. A
	high-pass filter can be calculated as the quadrature mirror filter of the
	low-pass filter. Daubechies wavelets can be defined by the scaling filter.
\end{description}

\subsection{Haar wavelet}
The Haar wavelet is a member of the Daubechies family of wavelets, based on the
work of the Belgian mathematician Ingrid Daubechies. The Daubechies wavelets is
a family of orthogonal wavelets defining a discrete wavelet transform. All
wavelets of the Daubechies family can be entirely defined by their scaling
filter. The Haar wavelet has a filter of length 2 and is therefore also
referenced to as the Daubechies-2 (D2) filter. It is the simplest wavelet of
the Daubechies family.

\paragraph{}
In 1910, the Hungarian mathematician Alfred Haar introduced Haar functions. The
Haar transform is one of the earliest examples of what is known now as a
compact, dyadic, orthonormal wavelet transform \cite{stankovic}. The Haar function
is the simplest and oldest orthonormal wavelet with compact support.

\paragraph{}
In this study and in the study by Karus \cite{karus2013}, the Haar filter is
used to perform wavelet transformations due to its simplicity and simple
interpretation.

\subsection{Discrete wavelet transformation using the Haar filter}
In each decomposition (e.g., each scaling or filtering step) the Haar function
adds more detail to the wavelets in the current level. The Haar filter captures
the differences between scale levels. The decomposition will be repeated until
the wavelets can no longer be divided. This happens when there is no more time
and/or frequency resolution left in the input signal to add further detail from
that signal.

\paragraph{}
Wavelet transformation using the Haar filter is widely used. For example, as a
way of digitalising an analogue signal in A-D converters, pattern recognition,
face recognition, image processing, data coding, multiplexing, digital
filtering, digital speech processing, voice controlled computing devices,
robotics, and compression mechanisms.

\begin{comment}
This chapter contains all the information needed to put the thesis into
context. It is common to use (a revised version) of your literature survey for
this purpose.
It is important to refer from your text to sources you have used, as listed in
your bibliography section (appendix). For example, “XP is a recent agile
development method [1]” is a common style of doing this, where the following
entry would be included in your bibliography:
[1] K. Beck, E. Gamma, Test infected: Programmers love writing tests, Java
Report 3 (7) (1998) 51–56.
If you want to refer to books you have read as part of the curriculum, you can
also do so in this way.
Have a look at Chapter 2 of this example thesis at Paul’s
homepage\footnote{http://homepages.cwi.nl/~paulk/thesesMasterSoftwareEngineering/2006/RichardKettelerij.pdf}.
\end{comment}