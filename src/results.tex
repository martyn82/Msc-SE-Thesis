\chapter{Results}
\label{results}

\section{Dead projects}
\label{section:deads}
The data set of 250 projects contains data points up to June 2013, because the
data was gathered in July 2013.

The data set contains 21 (8.4\%) dead projects. A total of 38 (15.2\%) projects
complied to the definition of a dead project as defined in section
\ref{def:dead}, but manual verification of the 38 potential dead projects, with
data up to April 2014, revealed that 21 (8.4\%) are \emph{really }\rm dead.

For each of the 38 projects, the project's website, source code repository, and
commit history was consulted. The results of the verification are shown in
Table \ref{table:deads}.

\begin{wraptable}{l}{40mm}
\caption{Dead projects}\label{table:deads}
\centering
\begin{tabular}{rr}
  \hline
 ID & Died at month \\ 
  \hline
317799 &   2 \\ 
  587198 &   3 \\ 
  588411 &   5 \\ 
  589515 &   7 \\ 
  587204 &   8 \\ 
  585077 &   9 \\ 
  587571 &  11 \\ 
  586805 &  14 \\ 
  360279 &  30 \\ 
  322065 &  37 \\ 
  11389 &  46 \\ 
  12053 &  68 \\ 
  3085 &  71 \\ 
  307140 &  71 \\ 
  4614 &  75 \\ 
  41745 &  80 \\ 
  155830 &  84 \\ 
  325178 &  92 \\ 
  4007 & 120 \\ 
  15700 & 121 \\ 
  12547 & 142 \\  
   \hline
\end{tabular}
\end{wraptable}

\paragraph{}
The other 17 projects complying to the definition of dead (section
\ref{def:dead}) appeared to be still alive. For 9 of the projects activity is
very low or rapidly decreasing; 5 projects were migrated to another source code
repository; and for 3 projects the activity is increasing after a long period of
no activity.

\paragraph{}
The 9 projects with very low or decreasing activity show signs of 'dying'. Their
community is slowly but surely abandoning the project as can be seen by a
decrease of 35\% to 55\% of contributors, and/or commits.

The 5 migrated projects for which the tracking information is not updated at
Ohloh are lost. The tracking is stopped from the moment the project is
migrated. The tracking and analysis can be recovered by updating the source
code locations at Ohloh, but for this study the project is out of sight.

The 3 projects that have had no activity in a year, but after that show little
increase of activity are more difficult to explain. Manual evaluation showed
that there are similar projects that eventually die, but there are also similar
projects that get 'resurrected'.

\paragraph{}
All, except one, of the projects in Table \ref{table:deads} are dead because it
was abandoned by the community of contributors. Except for project with ID
587204, which is still receiving updates, but at very slow and sporadic
intervals. However, since the updates do not involve code activity, it is
considered dead.

\paragraph{}
The first 7 projects have died before their first anniversary. The tail 14
projects died between 14 and 142 months of age.

\paragraph{}
When zooming in on the project with ID 317799, which is the youngest in this set
and died in its second month, it shows that this project has had a history of
the slightest change in its lines of code evolution. The project has had a total
number of 7 commits during the time it is being tracked by Ohloh (since
September 2011 up to now). A total of 4 contributors have worked on the project
since it was tracked. The most recent commit was done in October 2011.

\paragraph{}
Project with ID 12547 is the oldest. It died after 142 tracked months and has
had its most recent commits in February 2012. A total of 10 contributors have
worked on this project since May 2000.

\paragraph{}
The special case project, with ID 587204, is the only project that is not
abandoned by its (entire) community. The project has had 3 contributors over its
lifetime. One of them is still active every now and then. The most recent
commits were done in January 2014, the commits before that were in June 2012.
The commits involved updates in documentation, and the creation of a
configuration file for a continuous integration server. These commits do not
involve code changes.

The project died in July 2012, after 8 months since the first data point
tracked.

\section{Patterns}


\begin{comment}
- Factual results
- Tables and figures for clarification

This chapter presents and clarifies the results obtained during the research.
The focus should be on the factual results, not the interpretation or
discussion. Tables and graphics should be used to increase the clarity of the
results where applicable.
Have a look at the the results chapter in this example thesis on Paul’s
homepage\footnote{http://homepages.cwi.nl/~paulk/thesesMasterSoftwareEngineering/2006/ArnoldLankamp.pdf}.
\end{comment}
