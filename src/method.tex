\chapter{Research method}
\label{method}

This chapter elaborates on the steps to be taken in order to answer the
research questions. A description of each of the phases of the research is
given.

\section{Data selection}
\label{method:data}

The data for this research is provided by a tool by \citet{ohlohanalytics}.
This tool, ``\emph{OhlohAnalytics}\rm'', was developed as part of the
replicative research by \citet{bruntink2014}, and provides a validated and
cleansed data set of more than 10,000 OSS projects.

The projects are tracked by and gathered from
Ohloh\footnote{\url{http://www.ohloh.net}}: \emph{``Ohloh is a free, public
directory of Free and Open Source Software and the contributors who create and
maintain it.'' }\rm \cite{ohloh}.

At the time of this writing, Ohloh tracks more than 660,000 OSS projects varying
in all ranges of size, and popularity. The most popular projects currently being
Apache HTTP Server, Apache OpenOffice, Apache Subversion, Bash, Firebug, Linux
Kernel, Mozilla Firefox, MySQL, PHP, and Ubuntu.

\subsection{Data validation and cleansing}
The OhlohAnalytics tool provides us with an initial data set of more than
10,000 OSS projects \cite{bruntink2014}. The raw data gathered from Ohloh
regularly contains errors and inconsistencies. This tool analyses the
consistency of the data and cleanses the data where needed. This results in a
consistent data set of project evolution data.



\subsection{Project selection criteria}

From the initial 10,000+ projects of which the evolution data is consistent, a
subset of 250 projects will be made. \citet{karus2013} used a data set of 27
projects to perform his wavelet analysis research. We believe a larger
data set will yield more trustworthy results. The number 250 is a somewhat
arbitrary number, but chosen to have a substantial larger set than the initial
study by \citeauthor{karus2013}, but still keep it feasible to analyse and
validate the data and results within the given time constraints of 4 months for
this thesis.

\subsubsection{Subsequent data series}
Prior to the selection of the projects for the study, another validation
step is needed. Although the set of 10,000+ projects' evolution data is
consistent, probably not all of it is suited for wavelet analysis.

To be able to perform wavelet analysis on a series of data, we need to have
a subsequent series (i.e., without gaps) of evolution data for each project
under study. Therefore, all the 10,000+ projects will be analysed for continuity
and the projects that show gaps in the data are removed from the data set.

\subsubsection{Minimal sequence length}
A time series of 1 (monthly) data point is obviously not analysable over time
and uncomparable to larger projects. The threshold of at least 12 monthly data
points is chosen to minimise noise in the evolution data that may be caused by
too young or unstable projects.

\subsubsection{Representativeness}
Another criterion for the selection of projects in the data set is that they
form a subset that is representative to the set of all OSS projects tracked by
Ohloh. A tool created by \citet{nagappan} at Microsoft Research that is able
to determine the representativeness of a subset of projects will be used. This
tool scores projects by two metrics: total lines of code, and yearly
contributors count.

The tool by \citet{nagappan} is also capable of selecting a sample based on a
given baseline project and a total subset size. It keeps adding projects to the
subset as long as the project added will increase the total representativeness
of the sample subset to a given 'universe' of projects. We will use this tool
to select and score the sample subset of 250 projects using the Ohloh projects
as universe.



\section{Wavelet transform and analysis}
The analysis of the projects under study will be conducted in two main steps:
Wavelet transform, and Wavelet analysis. The second step is subdivided into two
smaller steps: Similar sequence identification, and Similar sequence grouping.

\subsection{Wavelet transform}
The first step in the analysis of time series of software evolution is the
wavelet transform. During this step discrete wavelet transform using the Haar
filter is applied on a project's signal (see also section
\ref{wavelet_analysis}). The results comprise the coefficients at each level of
decomposition and are saved for analysis.

\subsection{Wavelet analysis}
The wavelet analysis step is split into two sub-steps which will be described in
the following two subsections.

\subsubsection{Similar sequence identification}
During this step the coefficients from the wavelet transform step are
analysed to find \emph{similar sequences}\rm. A sequence is defined as a
subsequent series of coefficients - at a particular level of decomposition -
with a length between 3 and 65 points.
These thresholds were chosen to distinguish a sequence from an ordinary pair of
coefficients if the sequence is very short. On the contrary, if a sequence is
very long (larger than 65 points) we need to be able to distinguish the
sequence from the complete signal.

The 'points' mentioned here are not the same as data points in the data set.
The data points represent monthly data from a project, and the points in
sequences represent coefficients found at a certain level of decomposition
during the wavelet transform. More specifically, these 'points' are scaled or
filtered data points.

A sequence is considered 'similar' if at least one other sequence was found of
which the values are equal with respect to an allowed deviation. Similar
sequences may be found within one project, but preferrably be found across
projects.

\subsubsection{Similar sequence grouping}
In this step, the similar sequences from previous step are taken as input to
find 'patterns'.

A sequence is considered a 'pattern' if it appears at least 3 times in the
sequence data. Each pattern instance is assigned an identification number to be
used in further analysis.

The sequences that form a pattern instance are grouped together and added meta
data, such as, the number of occurrances, the list of projects having the
sequence, and whether it appears in dead, alive, or both dead and alive
projects.



\section{Pattern analysis}
In this phase of the research, the patterns will be analysed. The patterns found
in the previous phase will be validated mostly manually, in search of patterns
that are an objective warning sign [\ref{itm:question_warningsigns}].

\subsection{Warning signs}
A 'warning sign' is defined as a pattern found in a signal of a project's
evolution data (such as the value of the 'lines of code' metric over time), that
significantly increases the chances for that project to die (i.e., to end the
code evolution for that project).

\subsubsection{Definition of dead}
\label{def:dead}
A project is considered 'dead' if it complies to the following properties:
\begin{itemize}
	\item the project showed no change in LOC\footnote{LOC: abbreviation for
	'lines of code'.} for the last 12 months;
	\item the project has had 0 contributors in code for the last 12 months;
	\item the project has had 0 LOC churn\footnote{LOC churn: the sum of LOC added
	and LOC deleted.} for the last 12 months.
\end{itemize}

\noindent
The above properties combined specify that the project has had no code activity
for the past year, meaning the code evolution of the project has stopped.
According to this definition, changes in documentation, wiki pages, external
library updates, and other non-code changes are still allowed for dead
projects.

To compare dead and alive projects from the data set, we can use the above
definition to identify the projects from the data set that are dead, and the
other projects are still alive by definition.

\subsubsection{Dead project validation}
To be sure that the projects from our data set that comply to the definition of
dead are indeed dead, manual validation of these findings have to be done.
In verifying if a project is dead, we will consult the project's page on the
Ohloh website to see if the most recent data still complies to the definition of
dead. Furthermore, we will lookup the project's website and its repository to
find the latest commit history.

\subsection{Pattern selection}
Having the list of dead projects all patterns appearing in dead projects will be
selected for further analysis.

\begin{comment}
This section describes the methods used to answer the research questions. A
good structure of this section often follows the sub questions by providing a
method for each.

The research method can be based on the “Scientific method”, but more creative
solutions could be defined as well. In any case, the method needs a thorough
motivation grounded in theory in order to be acceptable.

As part of the method a number of hypotheses are described. These hypotheses
will be tested by the research, using the methods described here.

An important part of this section is validation. How will you evaluate and
validate the outcomes of the research? You can look at Paul Klint’s homepage
for examples of this section as
well\footnote{http://homepages.cwi.nl/~paulk/thesesMasterSoftwareEngineering/2006/RichardKettelerij.pdf}.
\end{comment}