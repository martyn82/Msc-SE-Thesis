\chapter{Research method}
\label{method}

\section{Wavelets in software evolution}
The use of wavelet transformation in the analysis of software evolution removes
the factor of project size and enables comparing projects equally to find
similar sequences.

\paragraph{}
In software evolution, the signal can be any measurable property of an evolving
entity, such as team size, lines of code, number of commits, etc.
These properties can be measured at a given time series, such as age in months,
or even a non-time-related series such as lines of code. Measuring these
properties over the evolution of a project gives a series of signals. Software
repositories are a source of signals in software evolution. This allows wavelet
transform be applied directly for mining software repositories.

\section{Warning signs}
\subsection{Dead projects}
The validation of the sequence identification was done by first classifying
a subset of the data set containing only dead projects. A dead project is
defined as a project that had no change in LOC in the past 12 months.\\

The list of dead projects is then used to filter the sequences detected by
these dead projects. The resulting set of sequences contains just the sequences
in dead projects. Each sequence is related to other sequences in other projects
which might be dead or not. These related projects are evaluated by checking if
they exist in the dead projects list.

\subsection{Dying projects}
The related projects that are not 'dead' are extracted for analysis. Their
related sequences will be analyzed. If a project from the 'not-dead' list has a
sequence related to that of a 'dead' project, and the related sequence occurs
at the end of the non-dead project's evolution data, then that project is a
good candidate to be classified as a 'dying' project.\\

The resulting data set contains a list of projects that might be 'dying', the
transform level at which the sequence was detected, and the number of times the
sequence was matched against another sequence. This set is of use when manually
validating the projects.

\section{Validation}
To validate whether the 'dying' projects are indeed dying, a manual
investigation has to be made. For the resulting projects, the Ohloh.net project
page can be consulted. The data set used contains data until July 2013,
therefore, it is easy to confirm if the trend of 'dying' has continued up to
April 2014. The activity report at Ohloh.net is very useful. It tells the
amounts of commits and contributors in the past 30 days and in the past 12
months. Changes that show a decrease of over 30\% in either commits,
contributors, or both are in general a bad sign.

\begin{comment}
This section describes the methods used to answer the research questions. A
good structure of this section often follows the sub questions by providing a
method for each.

The research method can be based on the “Scientific method”, but more creative
solutions could be defined as well. In any case, the method needs a thorough
motivation grounded in theory in order to be acceptable.

As part of the method a number of hypotheses are described. These hypotheses
will be tested by the research, using the methods described here.

An important part of this section is validation. How will you evaluate and
validate the outcomes of the research? You can look at Paul Klint’s homepage
for examples of this section as
well\footnote{http://homepages.cwi.nl/~paulk/thesesMasterSoftwareEngineering/2006/RichardKettelerij.pdf}.
\end{comment}