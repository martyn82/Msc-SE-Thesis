\chapter{Research method}
\label{method}

\section{Research approach}
The research will be conducted in the following steps:
\begin{description}
	\item[Literature study.] As part of the research plan, relevant literature will
	be studied in order to get familiar with the topics touched by this
	study. Important topics are, the ecosystem of OSS organisation and community,
	and the success and survival factors of OSS projects. Literature regarding
	wavelets, wavelet transforms and analysis, and in particular the Haar wavelet
	will be studied to understand its workings.

	\item[Data selection.] In this phase, the OSS projects for the study will be
	selected. The evolution data of these OSS projects will be validated and
	cleansed to ensure all data for these projects is consistent.

	Further evaluation of the data should be done prior to the selection of the
	data set for the study. It is expected that not all data is suitable for
	analysis.

	A criterion for the data that is expected to be important is that the
	evolution data for each project is continuous, in the sense that it does not
	contain 'gaps'. Analysis on time series is expected to require subsequent
	series of data. This expectation is formulated as a hypothesis
	(\ref{hyp:subsequent_data}) and will be tested for validity.
	
	The data set for analysis should also be representative to the total set of
	OSS projects in order to be able to generalise findings from the
	projects within the data set to the world of OSS projects in general.
	
	\item[Wavelet analysis.] This phase is divided into two sub-steps: wavelet
	transform, and wavelet analysis.
	
	First, the evolution data of the selected projects will be transformed using
	discrete wavelet transform (DWT). The results of these transforms are kept for
	further analysis. The signal is a model of a time variable and a frequency
	variable, and is also known as a 'wave'. The signal can be composed of any
	measurable property of a project's evolution, such as 'lines of code' (LOC).
	The choice for a suitable signal will be made.
	
	Second, the results of the transforms are being analysed to find patterns. A
	pattern is a sequence of transformed data that occurs multiple times. To be
	able to find patterns, a detailed definition of when a sequence is a
	pattern is needed. Choosing when a sequence becomes a pattern in terms of
	number of occurrances might influence the number of patterns found and
	possibly influence the types of patterns wavelet analysis is able to detect.
	
	\item[Pattern identification.] The patterns found in the Wavelet analysis step
	will be further analysed to find patterns that are 'warning signs'. This is a
	pattern that leads to the 'end of code evolution' for a project (recall
	\ref{itm:question_warningsigns}). The validation needed in this step is
	described in section \ref{method:validation}.
	
	\item[Analysis and conclusions.] The final phase in the study is the analysis
	of the results to come to conlusions and answers to the research questions
	(section \ref{questions}).
\end{description}

\section{Validation}
\label{method:validation}
\subsection{Dead projects}
\label{def:dead}
A project is considered 'dead' if it complies to the following properties:
\begin{enumerate}
	\item the project showed no change in 'lines of code' (LOC) for the last 12
	months;
	\item the project has had 0 contributors in code for the last 12 months;
	\item the project has had 0 LOC churn\footnote{LOC churn: the sum of LOC added
	and LOC deleted.} for the last 12 months.
\end{enumerate}

\noindent
The above properties combined specify that the project has had no code activity
for the past year, meaning the code evolution of the project has stopped.
According to this definition, changes in documentation, wiki pages, external
library updates, and other non-code changes are still allowed for dead
projects.

To compare dead and alive projects from the data set, the above definition can
be used to identify the projects from the data set that are dead. The projects
that are not 'dead' by this definition are considered to be 'alive'.

\paragraph{}
To be sure that the projects from our data set that comply to the definition of
dead (section \ref{def:dead}) are still dead beyond the data set, manual
validation of these findings have to be done.

In verifying if a project is dead, I will manually consult the project's
website, its source code repository, and possibly other sources to verify the
properties of a 'dead' project.

\subsection{Warning signs}
To recognise patterns that are warning signs, a selection of patterns occurring
in dead projects will be made for further analysis. The best candidates for
patterns being a warning sign are patterns occurring at the end of the
evolution of a dead project.

Knowing these patterns, selecting the projects having these patterns will be
done to find out if these projects have a higher chance of dying than projects
that do not show such pattern. An analysis of survival estimation will be done
to evaluate this (\ref{hyp:pattern_types}).

\section{Hypotheses}
In the search for patterns that may lead to the end of code evolution, the
following hypotheses are defined. During the research, these hypotheses will be
either confirmed or refuted.

\begin{description}
	\item[H1\label{hyp:subsequent_data}] \hspace{0em}
	The analysis of waveforms will need a subsequent data series in order
	to function properly.

	An investigation will be made to determine the differences between analysing
	subsequent data series and data series containing gaps. Whether this
	hypothesis is true will help answering \ref{itm:question_successfailure}.

	\item[H2\label{hyp:pattern_low_diff}] \hspace{0em}
	Patterns in LOC leading to end of code evolution contain low LOC differences.

	For all patterns found with LOC as signal, the maximum LOC difference will be
	computed and compared to patterns in LOC occurring at the end of code
	evolution in the dead projects.
	
	\item[H3\label{hyp:pattern_types}] \hspace{0em}
	Projects with a pattern occurring at the end of code evolution of a dead
	project have a higher chance of dying.

	The patterns occurring at the end of code evolution in dead projects are
	expected to be more likely a warning sign. A survival analysis of projects
	having these kinds of patterns will be conducted to confirm or refute this
	hypothesis.
\end{description}

\begin{comment}
- The plan
- Methodology / method per question
- Hypotheses
- Validation

This section describes the methods used to answer the research questions. A
good structure of this section often follows the sub questions by providing a
method for each.

The research method can be based on the “Scientific method”, but more creative
solutions could be defined as well. In any case, the method needs a thorough
motivation grounded in theory in order to be acceptable.

As part of the method a number of hypotheses are described. These hypotheses
will be tested by the research, using the methods described here.

An important part of this section is validation. How will you evaluate and
validate the outcomes of the research? You can look at Paul Klint’s homepage
for examples of this section as
well\footnote{http://homepages.cwi.nl/~paulk/thesesMasterSoftwareEngineering/2006/RichardKettelerij.pdf}.
\end{comment}