\chapter{Research method}
\label{method}

\section{Wavelet transform and analysis}
The word \emph{wavelet }\rm literally means 'small wave'. It comes from the
theory that a wave is continuous; e.g., it lasts from minus infinity to plus
infinity. The idea of a wavelet is that it is a sample of a wave that starts
and ends at some points in time.

\paragraph{}
A wavelet is a signal in a 2 dimensional space. Each dimension is a certain
domain. The abstract domains are the \emph{time domain }\rm and \emph{frequency
domain}\rm. In the case of an audio signal, the analogies are easily made. In
other fields, any observable property can serve as either frequency or time
value. As long as the relation between the two properties is functional within
the context of the signal.

\paragraph{}
Wavelet transformation is the sampling of a signal at different intervals which
gives a natural means of scaling the signal. Wavelet analysis is the analysis of
signals by decomposing the signal into wavelet coefficients (also known as shift
coefficients) and scaling coefficients (also known as filters).

The decomposition can be repeated on the scaling coefficients until the number
of resulting wavelet coefficients is smaller than the filter length.

\paragraph{}
\emph{Shifting }\rm is the operation of moving the wavelet through the time
domain. This operation is also known as \emph{translating }\rm the wavelet.
\emph{Filtering }\rm is the operation of scaling the wavelet in the frequency
domain. This operation is also known as \emph{dilating }\rm the wavelet.

\paragraph{}
The main advantages of wavelet transform in time series analysis are:
\begin{itemize}
	\item Scaling coefficients allow fuzzy matching as differences in details are
	``smoothed out''.
	\item Filter coefficients allow detection of small anomalies in series.
	\item Transform levels make series of different lengths or scale comparable.
\end{itemize}

\subsection{Haar filter}
In this study and in the study by Karus, the \emph{Daubechies }\rm filter of
length 2 (also known as \emph{Haar }\rm wavelet, due to its simplicity and
simple interpretation.

We apply discrete wavelet transform, meaning we use discrete shift when matching
the series with the wavelet. The scaling coefficients of discrete Haar wavelet
transform can be interpreted as a smoothed curve of the series. The wavelet
coefficients show temporal variations.

\paragraph{}
Naively speaken, the Haar transform is a means of digitalizing an analogue
signal. It still serves this purpose in many devices we use in a day-to-day
basis. The Haar transform sampling can be performed multiple times. The number
of times depends on the resolution of the analogue signal. The higher the
analogue resolution, the more sampling transforms are possible.\\

It starts by computing the integral of the full time series signal. It then
divides the signal into two equal parts over the time domain.
For each part it computes the integral. The difference of the two levels is
recorded. Thus, for each increase in detail (i.e., each re-division of the
signal) the extra information that is added is captured in that level. This
process can be repeated until there is insufficient time resolution to add more
information.\\

The result is a multi-layered sequence of values, where each layer represents a
decomposition level.

\subsection{Wavelets in software evolution}
The use of wavelet transformation in the analysis of software evolution removes
the factor of project size and enables comparing projects equally to find
similar sequences.

\paragraph{}
In software evolution, the signal can be any measurable property of an evolving
entity, such as team size, lines of code, number of commits, etc.
These properties can be measured at a given time series, such as age in months,
or even a non-time-related series such as lines of code. Measuring these
properties over the evolution of a project gives a series of signals. Software
repositories are a source of signals in software evolution. This allows wavelet
transform be applied directly for mining software repositories.


\begin{comment}
This section describes the methods used to answer the research questions. A
good structure of this section often follows the sub questions by providing a
method for each.

The research method can be based on the “Scientific method”, but more creative
solutions could be defined as well. In any case, the method needs a thorough
motivation grounded in theory in order to be acceptable.

As part of the method a number of hypotheses are described. These hypotheses
will be tested by the research, using the methods described here.

An important part of this section is validation. How will you evaluate and
validate the outcomes of the research? You can look at Paul Klint’s homepage
for examples of this section as
well\footnote{http://homepages.cwi.nl/~paulk/thesesMasterSoftwareEngineering/2006/RichardKettelerij.pdf}.
\end{comment}