\chapter{Research method}
\label{method}

\section{Wavelet analysis}
The research questions mentioned in section \ref{questions} are based on the
observations that wavelet analysis is used in other disciplines to analyze time
series. For example, wavelet analysis has been a successful method in economics
and (digital) signal processing (DSP).\\

Wavelet transformation is the sampling of a signal at different intervals which
gives a natural means of scaling the signal. Wavelet analysis is the analysis of
signals (time series) by decomposing the signal into wavelet coefficients (also
known as shift coefficients) and scaling coefficients (also known as filters).
The decomposition can be repeated on the scaling coefficients until the number
of resulting wavelet coefficients is smaller than the filter length.\\

In replicating the research by Karus we use the same Debauchies filter of length
2 (also known as Haar wavelet) due to its simplicity and simple interpretation.
We apply discrete wavelet transform, meaning we use discrete shift when matching
the series with the wavelet. The scaling coefficients of discrete Haar wavelet
transform can be interpreted as a smoothed curve of the series. The wavelet
coefficients show temporal variations.\\

\noindent
The main advantages of wavelet transform in time series analysis are:
\begin{itemize}
	\item Scaling coefficients allow fuzzy matching as differences in detals are
	``smoothed out''.
	\item Filter coefficients allow detection of small anomalies in series.
	\item Transform levels make series of different lengths or scale comparable.
\end{itemize}

\paragraph{}
In the case of software evolution, the signal can be any measurable property
of an evolving entity, such as team size, lines of code, number of commits, etc.
These properties can be measured at a given time series, such as age in days,
or even a non-time-related series such as lines of code. Measuring these
properties over the evolution of a project gives a series of signals. Software
repositories are a source of signals or time series in software evolution. This
allows wavelet transform be applied directly for mining software repositories.

\section{Data}
The original research was conducted by using a data set of 27 OSS projects. In
this study a data set of 250 OSS projects is used. The use of a larger data set
will in theory have a better spread of all OSS projects. Additionally, the
results of the research with a larger data set can be interpreted more
generally and with fewer noise.

\begin{comment}
In the original study by Karus it is shown that wavelet analysis can be used to
automatically identify evolutionary events in software development
\cite{karus2013}. He had found that by applying wavelet transformation on time
series, patterns can be found. Amongst different projects, similar patterns were
found. This indicates that these similar patterns could be the result of similar
events in the evolution of these projects.
\end{comment}

\begin{comment}
This section describes the methods used to answer the research questions. A
good structure of this section often follows the sub questions by providing a
method for each.
The research method can be based on the “Scientific method”, but more creative
solutions could be defined as well. In any case, the method needs a thorough
motivation grounded in theory in order to be acceptable.
As part of the method a number of hypotheses are described. These hypotheses
will be tested by the research, using the methods described here.
An important part of this section is validation. How will you evaluate and
validate the outcomes of the research? You can look at Paul Klint’s homepage
for examples of this section as
well\footnote{http://homepages.cwi.nl/~paulk/thesesMasterSoftwareEngineering/2006/RichardKettelerij.pdf}.
\end{comment}