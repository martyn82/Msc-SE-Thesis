\section{General information}
\subsection{Title}
Automatic Means of Detecting Warning Signs in Software Evolution

\subsection{Student name}
Martijn Florian Endenburg B ICT \\
Student number: 10442677

\subsection{Host organisation and group}
VNU Vacature Media (De Persgroep Nederland) \\
Department of Technology, Development \\
mt. Lincolnweg 40 \\
1033 SN Amsterdam \\
The Netherlands \\
\href{http://www.vnuvacaturemedia.nl}{www.vnuvacaturemedia.nl}

\subsection{Contact person}
Rutger Pannekoek \\
CTO at VNU Vacature Media \\
phone: 020 - 204 25 21 \\
email:
\href{mailto:r.pannekoek@vnuvacaturemedia.nl}{r.pannekoek@vnuvacaturemedia.nl}

\subsection{Summary}
\paragraph{}
The Master's project will be a replication of the paper \emph{``Automatic Means
of Identifying Evolutionary Events in Software Development''} by Siim Karus
(2013) \cite{karus2013}.

Karus has explored the usage of wavelet analysis to identify evolutionary events
in the evolution of open source software (OSS) projects. He has done the study
on a data set of 27 OSS repositories. This data set was chosen carefully to be a
representative set following the distribution of all OSS projects at the time of
the study. He selected alive, active, inactive, and dead projects.

Karus explored the usage of wavelet analysis on the evolution (time series) of
these projects to see if this analysis method can be used to find objective
warning signs in software projects leading to the end of code evolution.

\paragraph{}
According to Karus, knowing warning signs could help in:
\begin{itemize}
	\item Choosing OSS to implement in a business scenario.
	\item Making timely preparations for decommissioning an OSS project.
	\item Choosing a development process that best suites the aims of a software
	project for new OSS projects.
\end{itemize}

\paragraph{}
The main research question to be answered is:
\begin{description}
	\item[RQ:] \emph{\researchQuestion} \cite{karus2013}
\end{description}

\paragraph{}
As Karus keeps quite shallow about what kinds of events he was able to detect
and does not elaborate on this, I will use the methods studied by Karus to try
to find objective warning signs in code evolution and try to tell about the
kinds of events that we were able to detect.
