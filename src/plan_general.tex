\section{General information}
\subsection{Title}
Automatic Means of Detecting Warning Signs in Software Evolution

\subsection{Student name}
Martijn Florian Endenburg B ICT \\
Student number: 10442677

\subsection{Host organisation and group}
VNU Vacature Media (De Persgroep Nederland) \\
Department of Technology, Development \\
mt. Lincolnweg 40 \\
1033 SN Amsterdam \\
The Netherlands \\
\href{http://www.vnuvacaturemedia.nl}{www.vnuvacaturemedia.nl}

\subsection{Contact person}
Rutger Pannekoek \\
CTO at VNU Vacature Media \\
phone: 020 - 204 25 21 \\
email:
\href{mailto:r.pannekoek@vnuvacaturemedia.nl}{r.pannekoek@vnuvacaturemedia.nl}

\subsection{Summary}
\paragraph{}
The Master's project will be a replication of the paper \emph{``Automatic Means
of Identifying Evolutionary Events in Software Development''} by Siim Karus
(2013) \cite{karus2013}.

Karus has explored the usage of wavelet analysis to identify evolutionary events
(i.e., common patterns) in the evolution of open source software (OSS) projects.
The analysis was performed on a data set of 27 OSS projects. This data set was chosen
carefully to be a representative set following the distribution of all OSS
projects tracked by Ohloh.net at the time of the study. He selected active,
inactive, and dead projects.

However, Karus did not dive into the kinds of patterns he discovered. He did
conclude that similar patterns in software evolution can be identified by using
wavelet analysis, but he does not elaborate on these events.

In this Master's project I will extend the research done by Karus by using a
larger data set. At Ohloh.net, there are over 10,000 OSS projects available. It
should be possible to have a data set of at least 250 OSS projects for this
study.
Furthermore, I will dive into the detected events and try to know what they mean
in order to tell if an event is an objective warning sign. Additionally, it may
be possible to identify a warning sign in an early stage.

\paragraph{}
According to Karus, knowing warning signs could help in:
\begin{itemize}
	\item Choosing an OSS project to implement in a business scenario.
	\item Making timely preparations for decommissioning an OSS project.
	\item Choosing a development process that best suites the aims of a software
	project for new OSS projects.
\end{itemize}

\paragraph{}
The main research question to be answered is:
\begin{description}
	\item[RQ:] \emph{\researchQuestion} \cite{karus2013}
\end{description}

In order to answer the RQ, I will need to find answers to the following
questions:
\begin{description}
	\item[Q1:] \subQuestionOne
	\item[Q2:] \subQuestionTwo
\end{description}
