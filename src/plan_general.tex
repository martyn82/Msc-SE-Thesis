\section{General information}
\subsection{Title}
Automatic Means of Identifying Evolutionary Events in Software Development

\subsection{Student name}
Martijn Florian Endenburg

\subsection{Host organisation and group}
VNU Vacature Media (De Persgroep Nederland) \\
Department of Technology, Development \\
mt. Lincolnweg 40 \\
1033 SN, Amsterdam \\
The Netherlands

\subsection{Contact person}
Rutger Pannekoek \\
CTO at VNU Vacature Media \\
phone: 020 - 204 25 21 \\
email:
\href{mailto:r.pannekoek@vnuvacaturemedia.nl}{r.pannekoek@vnuvacaturemedia.nl}

\subsection{Summary}
\paragraph{}
This master’s project will be a replication of the paper \emph{Automatic Means
of Identifying Evolutionary Events in Software Development} by Siim Karus
(2013) \cite{karus2013}.

Karus has done this study on a data set of 27 open source software
repositories. This data set was chosen carefully to be representative to the
distribution of open source software projects that exist at the time of the
study; i.e., the relative amounts of dead, alive, and active projects.

Karus explored the usage of wavelet analysis on the evolution (time series) of
these projects to see if this analysis method can be used to find objective
warning signs in software projects leading to the end of code evolution.
