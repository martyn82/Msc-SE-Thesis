Many organisations rely on open-source software (OSS) projects to run their
businesses. Therefore, the need for the ability to assess an OSS project's
survival chances is increasing. However, the events influencing OSS projects
are mostly organic with little empirical validation.

\paragraph{}
This research is a replication of \textit{\repltitle} by \replauthor{} and
explores the use of wavelet analysis to automatically detect warning signs in
OSS projects, regardless of project size or scale. Knowing warning signs helps
in the decision to use, or whether to continue to use an OSS project.

\paragraph{}
Using wavelet analysis it is possible to detect events that lead to the end of
code evolution of an OSS project. This study presents an analysis of the
survivability of projects having such warning signs, and under what conditions
wavelet analysis succeeds or fails in detecting warning signs.

\keywords{open-source software, software analytics, software evolution, project
survivability, warning signs, wavelet analysis}

\begin{comment}
1. State the problem.
2. State why the problem is a problem.
3. Startling.
4. State the implication of the startling.

See:
Ralph E. Johnson, Kent Beck, Grady Booch, William R. Cook, Richard P. Gabriel,
and Rebecca Wirfs-Brock.
How to Get a Paper Accepted at OOPSLA.
In Timlynn Babitsky and Jim Salmons, editors,
Proceedings of the Eighth Annual Conference on Object-Oriented Programming Systems, Languages and Applications, OOPSLA,
pages 429–436. ACM, 1993.
\end{comment}
